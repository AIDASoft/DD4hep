%=============================================================================
\documentclass[10pt,a4paper]{article}
%
%
%
\usepackage{graphicx}
\usepackage{hyperref}
\usepackage{verbatim}
\usepackage{fix-cm}
\usepackage{lineno}
\usepackage{fancyhdr}
%
%
\oddsidemargin  0.1 in
\evensidemargin 0.1 in
%
%
\newlength{\backindent}\setlength{\backindent}{2cm}
\textwidth 5.375 in % Width of text line.
\advance\textheight by1.4cm
\advance\voffset by-1.4cm
\advance\textwidth by\backindent
%
%
% === Fancy headers setup  ===============================
%
\setlength{\headheight}{15.2pt}
\pagestyle{fancyplain} {
\fancyhead[L]{\includegraphics[height=10mm]{DD4hep-AIDA-logo.png}\vspace{-0.3cm}}
\fancyhead[C]{}
\fancyhead[R]{\sffamily{\underline{\hspace{6cm}Advanced European Infrastructures for Detectors at Accelerators}}}
\fancyfoot[L]{}
\fancyfoot[C]{\sffamily{User Manual}}
\fancyfoot[R]{\sffamily{\thepage}}
}
%
%
\newcommand{\tw}[1]{${\tt{#1}}$}
\newcommand{\tts}[1]{{\tt\small{#1}}}
\newcommand{\bold}[1]{{\bf{#1}}}
%
%
\newcommand{\DDE}{{$\tt{DDEve}$\space}}
\newcommand{\DDhep}{{$\tt{DD4hep}$\space}}
\newcommand{\DDH}{{$\tt{DD4hep}$\space}}
\newcommand{\DDG}{{\tt{DDG4}\space}}
\newcommand{\DDA}{{\tt{DDAlign}\space}}
\newcommand{\DDR}{{\tt{DDRec}\space}}
%%
%
\newcommand{\docline}[2]{\vspace{0.1cm}{\bf{#1}} & \parbox{14.5cm}{#2}\\}
%
% === Specialization of the lineno package
%
\renewcommand{\linenumberfont} {\normalfont\small\sffamily}
\renewcommand{\makeLineNumber} {\makeLineNumberLeft}
\renewcommand{\linenumbersep} {2pt}
%
% === Set font to code section with line numbers
%
\newenvironment{code}{\par\vspace{0.01cm}\small\linenumbers\verbatim\setcounter{linenumber}{1}}{\endverbatim\nolinenumbers\vspace{-0.02cm}}%
%
% === Set font to code section with line numbers
%
\newenvironment{unnumberedcode}{\par\vspace{-0.1cm}\small\verbatim\setcounter{linenumber}{1}}%
{\endverbatim\vspace{-0.2cm}}
%
% === Command to insert http links to the DD4hep geomtery package
%
\newcommand{\detdesc}[2]
{
    \href{http://www.cern.ch/frankm/DD4hep/doc/#1}{#2}
}
%
% === Command to insert http links to the ROOT geomtery package
%
\newcommand{\tgeo}[2]
{
    \href{http://root.cern.ch/root/html/#1.html}{#2}
}
%
% ===  Compactify the item list  =========================
%
\newcommand{\itemcompact}{\setlength{\itemsep}{1pt}\setlength{\parskip}{0pt}\setlength{\parsep}{0pt}}
%
%
% ===  Title page command  ===============================
%
\newcommand{\mytitle}[3]{
\begin{titlepage}
%
\pagestyle{empty}
%
\includegraphics[height=25mm] {DD4hep-AIDA-logo.png}

\vspace{0.02cm}

{\sffamily{\underline{\hspace{6cm}Advanced European Infrastructures for Detectors at Accelerators}}}

\vspace{2cm}

\begin{center}
{\fontsize{72}{32}\selectfont{\bfseries{#1}}}

\vspace{3cm}
{\Huge\bf{#2}}

\vspace{5cm}
{#3}
%M.Frank%\textsuperscript{1},
%F.Gaede\textsuperscript{2},
%C.Grefe\textsuperscript{1},
%P.Mato\textsuperscript{1}
%{\textsuperscript{1} 
%{CERN, 1211 Geneva 23, Switzerland}
%{\textsuperscript{2} Desy, 22607 Hamburg, Germany}
\end{center}
\end{titlepage}
}

%
\pagestyle{fancyplain}{\fancyfoot[C]{\sffamily{DDG4 User Manual}}}
%
\begin{document}   
%
\mytitle{
DDG4
}{
A Simulation Toolkit for \\
\vspace{0.5cm}
High Energy Physics Experiments\\
\vspace{0.5cm}
using Geant4 and the \\
\vspace{0.5cm}
DD4hep Geometry Description
}{DDG4 User Manual}
%
%
%==  Abstract  ===============================================================
\pagestyle{plain}
\pagenumbering{Roman}
\setcounter{page}{1}
\begin{abstract}
%=============================================================================

\noindent
\normalsize
Simulating the detector response is an essential tool in high energy physics
to analyze the sensitivity of an experiment to the underlying physics.
Such simulation tools require a detailed though convenient detector description as 
it is provided by the \DDhep toolkit.
We will present the generic simulation toolkit \DDG using the \DDhep detector 
description toolkit. 
The toolkit implements a modular and flexible approach to simulation activities
using Geant4. User defined simulation applications using \DDG 
can easily be configured, extended using specialzed action routines.
The design is strongly driven by easy of use;
developers of detector descriptions and applications using
them should provide minimal information and minimal specific
code to achieve the desired result.

\end{abstract}

\vspace{8cm}

\begin{center}
{\large{\bf{
\begin{tabular} {| l | l | l |}
\hline
\multicolumn{3}{| c |}{} \\[0.2cm]
\multicolumn{3}{| c |}{Document History} \\[0.2cm]
\multicolumn{3}{| c |}{} \\[0.2cm]
\hline
                 &      &        \\
Document         &      &        \\
version          & Date & Author \\[0.2cm] \hline
                 &      &        \\
1.0              & 19/11/2013 & Markus Frank CERN/LHCb  \\
                 &      &        \\        \hline 
\end{tabular}
}}}
\end{center}

\clearpage
%
%
%==  TOC  ====================================================================
\tableofcontents
\clearpage
%
%
%=============================================================================
% Manual
%=============================================================================
\pagenumbering{arabic}
\setcounter{page}{1}

%=============================================================================
\section{Introduction}
\label{sec:ddg4-user-manual-introduction}
%=============================================================================
\noindent
This manual should introduce to the DDG4 framework. 
One goal of \DDG is to easily configure the simulation applications
capable of simulating the physics response of detector configurations 
as they are used for example in high energy physics experiments.
In such simulation programs the user normally has to define the 
experimental setup in terms of its geometry and in terms of its 
active elements which sample the detector response.

\noindent
The goal of \DDG is to generalize the configuration of a simulation
application to a degree, which does not force users to write code
to test a detector design. At the same time it should of course
be feasible to supply specialized user written modules which are supposed
to seamlessly operate together with standard modules supplied by the toolkit.
Detector-simulation depends strongly on the use of an underlying simulation
toolkit, the most prominent candidate nowadays being Geant4~\cite{bib:geant4}.
\DDhep supports simulation activities with Geant4 providing
an automatic translation mechanism between geometry representations.
The simulation response in the active elements of the detector
is strongly influenced by the technical 
choices and precise simulations depends on the very specific detection techniques.

\noindent
Similar to the aim of \DDhep~\cite{bib:DD4hep}, 
where with time a standard palette of detector
components developed by users should become part of the toolkit,
\DDG also hopes to provide a standard palette of components used
to support simulation activities for detector layouts
where detector designers may base the simulation of a planned experiment 
on these predefined components for initial design and optimization 
studies. The longterm vision is to construct simulation applications
writing only new components not yet present i.e. the main work will be to
select the appropriate components from the palette and connect them
to a functional program.

\noindent
This is not a manual to Geant4 nor the basic infrastructure of \DDhep.
It is assumed that this knowledge is present and the typical glossary 
is known.

%=============================================================================
\section{The Geant4 User Interface}
\label{sec:ddg4-user-manual-geant4-interface}
%=============================================================================

\noindent
The Geant4 simulation toolkit~\cite{bib:geant4} implements a very complex
machinery to simulate the energy deposition of particles traversing materials.
To ease its usage for the clients and to shield clients from the complex
internals when actually implementing a simulation applications for a 
given detector design, it provides several user hooks
as shown in Figure~\ref{fig:ddg4-g4runmanager-anatomy}. Each of these hooks 
serves a well specialized purpose, but unfortunately also leads to very 
specialized applications. One aim of \DDG is to formalize these user 
actions so that the invocation at the appropriate time may be purely
data driven.
\begin{figure}[h]
  \begin{center}
    \includegraphics[height=70mm] {DDG4-G4RunManagerAnatomy.png}
    \caption{The various user hooks provided by Geant4. Not shown here
              is the callback system interfacing to the active elements
              of the detector design.}
    \label{fig:ddg4-g4runmanager-anatomy}
  \end{center}
\end{figure}

\noindent
In detail the following object-hooks allow the client to define user provided actions:
\begin{itemize}\itemcompact
\item The \bold{User Physics List} allows the client to customize and define 
    the underlying physics process(es) which define the particle interactions 
    inside the detector defined with the geometry description.
    These interactions define the detector response in terms of 
    energy depositions.
\item The \bold{Run Action} is called once at the start and end of a run. 
    i.e. a series of generated events. These two callbacks
    allow clients to define run-dependent actions such as statistics
    summaries etc.
\item The \bold{Primary Generator Action} is called for every event.
    During the callback all particles are created which form the 
    microscopic kinematic action of the particle collision.
    This input may either origin directly from an event generator program
    or come from file.
\item The \bold{Event Action} is called once at the start and the end of each event.
     It is typically used for a simple analysis of the processed event.
     If the simulated data should be written to some persistent medium, 
     the call at the end of the event processing is the appropriate place.
\item The \bold{Tracking Action} 
\item The \bold{Stepping Action} 
\item The \bold{Stacking Action} 
\end{itemize}
\noindent
Geant4 provides all callbacks with the necessary information in the form of 
appropriate arguments.

\noindent
Besides the callback system, Geant4 provides callbacks whenever a particle
traverses a sensitive volume. These callbacks are called 
- similar to event actions - once at the start and the end of the event,
but in addition, if either the energy deposit of a particle in the 
sensitive volume exceeds some threshold. The callbacks are formalized within 
the base class \tts{G4VSensitiveDetector}.

%=============================================================================
\section{DDG4 Implementation}
\label{sec:ddg4-user-manual-implementation}
%=============================================================================

\noindent
A basic design criteria of the a \DDG simulation application was to 
process any user defined hook provided by Geant4 as a series of algorithmic
procedures, which could be implemented either using inheritance or by 
a callback mechanism registering functions fulfilling a given signature.
Such sequences are provided for all actions mentioned in the list in 
Section~\ref{sec:ddg4-user-manual-geant4-interface} as well as for 
the callbacks to sensitive detectors.

\noindent
The callback mechanism was introduced to allow for weak coupling between 
the various actions. For example could an action performing monitoring
using histograms at the event level initialize or reset its histograms
at the start/end of each run. To do so, clearly a callback at the 
start/end of a run would be necessary.

\noindent
In the following sections a flexible and extensible interface to hooks
of Geant4 is discussed starting with the description of the basic
components \tts{Geant4Kernel} and \tts{Geant4Action} followed by the 
implementation of the relevant specializations.
The specializations exposed are sequences of such actions,
which also call registered objects.
In later section the configuration and the combination of these components 
forming a functional simulation application is presented.

%=============================================================================
\subsection{The Application Core Object: \tts{Geant4Kernel}}
\label{sec:ddg4-user-manual-implementation-geant4kernel}
%=============================================================================

\noindent
The kernel object is the central context of a \DDG simulation application and
gives all clients access to the user hooks (see Figure~\ref{fig:ddg4-geant4-kernel}).
All Geant4 callback structures are exposed so that clients can easily 
objects implementing the required interface or register callbacks with the 
correct signature.
\begin{figure}[h]
  \begin{center}
    \includegraphics[height=80mm] {DDG4-Geant4Kernel.png}
    \caption{The sensitive detector design.}
    \label{fig:ddg4-geant4-kernel}
  \end{center}
\end{figure}

%=============================================================================
\subsection{The Base Class of DDG4 Actions: \tts{Geant4Action}}
\label{sec:ddg4-user-manual-implementation-geant4action-base}
%=============================================================================

\noindent
The class \tts{Geant4Action} is a common component interface providing 
the basic interface to the framework to
\begin{itemize}\itemcompact
\item configure the component using a property mechanism
\item provide an appropriate interface to Geant4 interactivity. The interactivity 
    included a generic way to change and access properties from the Geant4 UI 
    prompt as well as executing registered commands.
\item As shown in Figure~\ref{fig:ddg4-implementation-geant4-action}, the 
    base class also provides to its sub-class a reference to the \tts{Geant4Kernel}
    objects through the \tts{Geant4Context}.
\end{itemize}
The \tts{Geant4Action} is a named entity and can be uniquely identified within
a sequence attached to one Geant4 user callback.
%=============================================================================
\begin{figure}[h]
  \begin{center}
    \includegraphics[height=30mm] {DDG4-Geant4Action.png}
    \caption{The design of the common base class \tts{Geant4Action}.}
    \label{fig:ddg4-implementation-geant4-action}
  \end{center}
\end{figure}

\noindent
\DDG knowns two types of actions: global actions and anonymous actions.
Global actions are accessible externally from the \tts{Geant4Kernel} instance.
Global actions are also re-usable and hence may be contribute to several 
action sequences (see the following chapters for details). Global actions 
are uniquely identified by their name.
Anonymous actions are known only within one sequence and normally
are not shared between sequences.

%=============================================================================
\subsubsection{The Properties of \bold{Geant4Action} Instances}
\label{sec:ddg4-implementation-geant4-action-properties}
%=============================================================================

\noindent
Nearly any subclass of a \tts{Geant4Action} needs a flexible configuration 
in order to be reused, modified etc. The implementation of the mechanism
uses a very flexible value conversion mechanism using \tts{boost::spirit},
which support also conversions between unrelated types provided a dictionary 
is present.

\noindent
Properties are supposed to be member variables of a given action object.
To publish a property it needs to be declared in the constructor as shown here:
\begin{unnumberedcode}
  declareProperty("OutputLevel", m_outputLevel = INFO);
  declareProperty("Control",     m_needsControl = false);
\end{unnumberedcode}
The internal setup of the \tts{Geant4Action} objects then ensure that 
all declared properties will be set after the object construction to the 
values set in the setup file.

\noindent
\bold{Note:} Because the values can only be set \bold{after} the object 
was constructed, the actual values may not be used in the constructor
of any base or sub-class.

%=============================================================================
\subsection{Geant4 Action Sequences}
\label{sec:ddg4-user-manual-implementation-geant4action-sequences}
%=============================================================================

\noindent
The main action sequences have a fixed name. These are
\begin{itemize}

\item The \bold{RunAction} attached to the \tts{G4UserRunAction}, implemented 
    by the \tts{Geant4RunActionSequence} class and is called at the start and the end of 
    every run (beamOn). Members of the \tts{Geant4RunActionSequence} are of type
    \tts{Geant4RunAction} and receive the callbacks by overloading the two routines:
\begin{unnumberedcode}
/// begin-of-run callback
virtual void begin(const G4Run* run);
/// End-of-run callback
virtual void end(const G4Run* run);
\end{unnumberedcode}
    or register a callback with the signature {\tts{void (T::*)(const G4Run*)}}
    either to receive begin-of-run or end-or-calls using the methods:
\begin{unnumberedcode}
/// Register begin-of-run callback. Types Q and T must be polymorph!
template <typename Q, typename T> void callAtBegin(Q* p, void (T::*f)(const G4Run*));
/// Register end-of-run callback. Types Q and T must be polymorph!
template <typename Q, typename T> void callAtEnd(Q* p, void (T::*f)(const G4Run*));
\end{unnumberedcode}
    of the \tts{Geant4RunActionSequence} from the \tts{Geant4Context} object.


\item The \bold{EventAction} attached to the \tts{G4UserEventAction}, implemented 
    by the \tts{EventActionSequence} class and is called at the start and the end of 
    every event. Members of the \tts{Geant4EventActionSequence} are of type
    \tts{Geant4EventAction} and receive the callbacks by overloading the two routines:
\begin{unnumberedcode}
/// Begin-of-event callback
virtual void begin(const G4Event* event);
/// End-of-event callback
virtual void end(const G4Event* event);
\end{unnumberedcode}
    or register a callback with the signature {\tts{void (T::*)(const G4Event*)}}
    either to receive begin-of-run or end-or-calls using the methods:
\begin{unnumberedcode}
/// Register begin-of-event callback
template <typename Q, typename T> void callAtBegin(Q* p, void (T::*f)(const G4Event*));
/// Register end-of-event callback
template <typename Q, typename T> void callAtEnd(Q* p, void (T::*f)(const G4Event*));
\end{unnumberedcode}
    of the \tts{Geant4EventActionSequence} from the \tts{Geant4Context} object.


\item The \bold{GeneratorAction} attached to the \tts{G4VUserPrimaryGeneratorAction}, implemented 
    by the \tts{Geant4GeneratorActionSequence} class and is called at the start of 
    every event and provided all initial tracks from the Monte-Carlo generator.
    Members of the \tts{Geant4GeneratorActionSequence} are of type
    \tts{Geant4EventAction} and receive the callbacks by overloading the member function:
\begin{unnumberedcode}
/// Callback to generate primary particles
virtual void operator()(G4Event* event);
\end{unnumberedcode}
    or register a callback with the signature {\tts{void (T::*)(G4Event*)}}
    to receive calls using the method:
\begin{unnumberedcode}
/// Register primary particle generation callback.
template <typename Q, typename T> void call(Q* p, void (T::*f)(G4Event*));
\end{unnumberedcode}
    of the \tts{Geant4GeneratorActionSequence} from the \tts{Geant4Context} object.

\end{itemize}
\begin{figure}[t]
  \begin{center}
    \includegraphics[width=160mm] {DDG4-TrackingAction.png}
    \caption{The design of the tracking action sequence. Specialized 
               tracking action objects inherit from the \tts{Geant4TrackingAction}
               object and must be attached to the sequence.}
    \label{fig:ddg4-implementation-tracking-action}
  \end{center}
\end{figure}

\begin{itemize}
\item The \bold{TrackingAction} attached to the \tts{G4UserTrackingAction}, 
    implemented by the \tts{Geant4-} \tts{Tracking\-Action\-Sequence} class 
    and is called at the start and the end of tracking one single particle 
    trace through the material of the detector.
    Members of the \tts{Geant4\-Tracking\-ActionSequence} are of type
    \tts{Geant4TrackingAction} and receive the callbacks by overloading the member function:
\begin{unnumberedcode}
/// Pre-tracking action callback
virtual void begin(const G4Track* trk);
/// Post-tracking action callback
virtual void end(const G4Track* trk);
\end{unnumberedcode}
    or register a callback with the signature {\tts{void (T::*)(const G4Step*, G4SteppingManager*)}}
    to receive calls using the method:
\begin{unnumberedcode}
/// Register Pre-track action callback
template <typename Q, typename T> void callAtBegin(Q* p, void (T::*f)(const G4Track*));
/// Register Post-track action callback
template <typename Q, typename T> void callAtEnd(Q* p, void (T::*f)(const G4Track*));
\end{unnumberedcode}


\item The \bold{SteppingAction} attached to the \tts{G4UserSteppingAction}, implemented 
    by the \tts{Geant4-} \tts{SteppingActionSequence} class and is called for each
    step when tracking a particle.
    Members of the \tts{Geant4SteppingActionSequence} are of type
    \tts{Geant4SteppingAction} and receive the callbacks by overloading the member function:
\begin{unnumberedcode}
/// User stepping callback
virtual void operator()(const G4Step* step, G4SteppingManager* mgr);
\end{unnumberedcode}
    or register a callback with the signature {\tts{void (T::*)(const G4Step*, G4SteppingManager*)}}
    to receive calls using the method:
\begin{unnumberedcode}
/// Register stepping action callback.
template <typename Q, typename T> void call(Q* p, void (T::*f)(const G4Step*, 
                                                               G4SteppingManager*));
\end{unnumberedcode}


\item The \bold{StackingAction} attached to the 
    {\tts{G4UserStackingAction}}, implemented by the \tts{Geant4-}\\
    \tts{StackingActionSequence} class.
    Members of the \tts{Geant4StackingActionSequence} are of type\\
    \detdesc{html/class_d_d4hep_1_1_simulation_1_1_geant4_stacking_action.html}
    {\tts{Geant4StackingAction}} and receive the callbacks by overloading the member functions:
\begin{unnumberedcode}
/// New-stage callback
virtual void newStage();
/// Preparation callback
virtual void prepare();
\end{unnumberedcode}
    or register a callback with the signature {\tts{void (T::*)()}}
    to receive calls using the method:
\begin{unnumberedcode}
/// Register begin-of-event callback. Types Q and T must be polymorph!
template <typename T> void callAtNewStage(T* p, void (T::*f)());
/// Register end-of-event callback. Types Q and T must be polymorph!
template <typename T> void callAtPrepare(T* p, void (T::*f)());
\end{unnumberedcode}
\end{itemize}

\noindent
All sequence types support the method \tts{void adopt(T* member\_reference)}
to add the members. Once adopted, the sequence takes ownership and manages
the member. The design of the sequences is very similar. 
Figure~\ref{fig:ddg4-implementation-tracking-action} show as an example 
the design of the \tts{Geant4TrackingAction}.

\begin{figure}[h]
  \begin{center}
    \includegraphics[height=110mm] {DDG4-Sensitive-detector.png}
    \caption{The sensitive detector design. The actual energy deposits are 
        collected in user defined subclasses of the \tts{Geant4Sensitive}.
        Here, as an example possible actions called \tts{TrackerHitCollector},
        \tts{TrackerDetailedHitCollector} and \tts{TrackerHitMonitor} are shown.}
    \label{fig:ddg4-implementation-sensitive-detector}
  \end{center}
\end{figure}

%=============================================================================
\subsection{Sensitive Detectors}
\label{sec:ddg4-user-manual-geant4sensitivedetectors}
%=============================================================================

\noindent
Sensitive detectors are associated by the detector designers to all active 
materials, which would produce a signal which can be read out. In Geant4 this concept
is realized by using a base class \tts{G4VSensitiveDetector}, which receives 
a callback at the begin and the end of the event processing and at each step
inside the active material whenever an energy deposition occurred.

\noindent
The sensitive actions do not necessarily deal only the collection of energy 
deposits, but could also be used to simply monitor the performance of the
active element e.g. by producing histograms of the absolute value or the 
spacial distribution of the depositions.

\noindent
Within \DDG the concept of sensitive  detectors is implemented as a
configurable  action sequence of type 
\detdesc{html/class_d_d4hep_1_1_simulation_1_1_geant4_sens_det_action_sequence.html}
{\tts{Geant4SensDetActionSequence}}
calling members of the type 
\detdesc{html/struct_d_d4hep_1_1_simulation_1_1_geant4_sensitive.html}
{\tts{Geant4Sensitive}} as shown in 
Figure~\ref{fig:ddg4-implementation-sensitive-detector}. The actual processing
part of such a sensitive action is only called if the and of a set of
required filters of type \tts{Geant4Filter} is positive (see also 
section~\ref{sec:ddg4-implementation-sensitive-detector-filters}). No filter 
is also positive. Possible filters are e.g. particle filters, which ignore the
sensitive detector action if the particle is a \tts{geantino} or if the
energy deposit is below a given threshold.

\noindent
Objects of type \tts{Geant4Sensitive} receive the callbacks by overloading the 
member function:
\begin{unnumberedcode}
  /// Method invoked at the begining of each event.
  virtual void begin(G4HCofThisEvent* hce);
  /// Method invoked at the end of each event.
  virtual void end(G4HCofThisEvent* hce);
  /// Method for generating hit(s) using the information of G4Step object.
  virtual bool process(G4Step* step, G4TouchableHistory* history);
  /// Method invoked if the event was aborted.
  virtual void clear(G4HCofThisEvent* hce);
\end{unnumberedcode}
or register a callback with the signature {\tts{void (T::*)(G4HCofThisEvent*)}}
respectively {\tts{void (T::*)(G4Step*, G4TouchableHistory*)}} 
to receive callbacks using the methods:
\begin{unnumberedcode}
  /// Register begin-of-event callback
  template <typename T> void callAtBegin(T* p, void (T::*f)(G4HCofThisEvent*));
  /// Register end-of-event callback
  template <typename T> void callAtEnd(T* p, void (T::*f)(G4HCofThisEvent*));
  /// Register process-hit callback
  template <typename T> void callAtProcess(T* p, void (T::*f)(G4Step*, G4TouchableHistory*));
  /// Register clear callback
  template <typename T> void callAtClear(T* p, void (T::*f)(G4HCofThisEvent*));
\end{unnumberedcode}

%=============================================================================
\subsubsection{Sensitive Detector Filters}
\label{sec:ddg4-implementation-sensitive-detector-filters}
%=============================================================================

\noindent
Filters are called by Geant4 before the
hit processing in the sensitive detectors start. The global filters
may be shared between many sensitive detectors. Alternatively filters
may be directly attached to the sensitive detector in question.
Attributes are directly passed as properties to the filter action.

\newpage
%=============================================================================
\subsection{The Geant4 Physics List}
\label{sec:ddg4-implementation-physics-list}
%=============================================================================
\noindent 
Geant4 provides the base class \tts{G4VUserPhysicsList}. Any user defined
physics list must provide this interface. DDG4 provides such an interface
through the ROOT plugin mechanism using the class \tts{G4VModularPhysicsList}.
The flexibility of \DDG allows for several possibilities to setup the Geant4
physics list.
\begin{itemize}
\item The \bold{physics list} may be configured as a sequence of type 
    \detdesc{html/class_d_d4hep_1_1_simulation_1_1_geant4_physics_list_action_sequence.html}
    {\tts{Geant4PhysicsListActionSequence}}.
    Members of the \detdesc{html/class_d_d4hep_1_1_simulation_1_1_geant4_physics_list_action_sequence.html}
    {\tts{Geant4PhysicsListActionSequence}} are of type
    \detdesc{html/class_d_d4hep_1_1_simulation_1_1_geant4_physics_list.html}
    {\tts{Geant4PhysicsList}} and receive the callbacks by overloading 
    the member functions:
\begin{unnumberedcode}
  /// Callback to construct the physics constructors
  virtual void constructProcess(Geant4UserPhysics* interface);
  /// constructParticle callback
  virtual void constructParticles(Geant4UserPhysics* particle);
  /// constructPhysics callback
  virtual void constructPhysics(Geant4UserPhysics* physics);
\end{unnumberedcode}
    or register a callback with the signature {\tts{void (T::*)(Geant4UserPhysics*)}}
    to receive calls using the method:
\begin{unnumberedcode}
  /// Register process construction callback t
  template <typename Q, typename T> void constructProcess(Q* p, void (T::*f)(Geant4UserPhysics*));
  /// Register particle construction callback
  template <typename Q, typename T> void constructParticle(Q* p, void (T::*f)(Geant4UserPhysics*));
\end{unnumberedcode}
    The argument of type \detdesc{html/class_d_d4hep_1_1_simulation_1_1_geant4_user_physics.html}
    {\tts{Geant4UserPhysics}} provides a basic interface to the original
    \tts{G4VModular}- \tts{PhysicsList}, which allows to register physics constructors etc.

\item In most of the cases the above approach is an overkill and often even too flexible.
    Hence, alternatively, the physics list may consist of a single entry of type 
    \detdesc{html/class_d_d4hep_1_1_simulation_1_1_geant4_physics_list.html}
    {\tts{Geant4PhysicsList}}.
\end{itemize}

\noindent
The basic implementation of the \tts{Geant4PhysicsList} supports the usage of various
\begin{itemize}\itemcompact
\item \detdesc{}{particle constructors}, such as single particle constructors like   
    \tts{G4Gamma} or \tts{G4Proton}, or whole particle groups like
    \tts{G4BosonConstructor} or \tts{G4IonConstrutor},
\item \detdesc{}{physics process constructors}, such as e.g. \tts{G4GammaConversion},
    \tts{G4PhotoElectricEffect} or\\ \tts{G4ComptonScattering}, 
\item \detdesc{}{physics constructors} combining particles and the corresponding 
    interactions, such as\\ e.g. \tts{G4OpticalPhysics},
    \tts{HadronPhysicsLHEP} or \tts{G4HadronElasticPhysics} and
\item \detdesc{}{predefined Geant4 physics lists}, such as \tts{FTFP\_BERT},
    \tts{CHIPS} or \tts{QGSP\_INCLXX}. This option is triggered by the 
    content of the string property "extends" of the \tts{Geant4Kernel::physicsList()} action.
\end{itemize}
These constructors are internally connected to the above callbacks to register themselves. 
The constructors are instantiated using the ROOT plugin mechanism.

\noindent
The description of the above interface is only for completeness. The basic idea is,
that the physics list with its particle and physics constructors is configured
entirely data driven using the setup mechanism described in the following
chapter. However, DDG4 is not limited to the data driven approach. Specialized 
physics lists may be supplied, but there should be no need.
New physics lists could always be composed by actually providing new physics
constructors and actually publishing these using the factory methods:
\begin{code}
// Framework include files
#include "DDG4/Factories.h"

#include "My_Very_Own_Physics_Constructor.h"
DECLARE_GEANT4_PHYSICS(My_Very_Own_Physics_Constructor)
\end{code}
where \tts{My\_Very\_Own\_Physics\_Constructor} represents a sub-class of
\tts{G4VPhysicsConstructor}.

\newpage
%=============================================================================
\subsection{The Support of the Geant4 UI: \tw{Geant4UIMessenger}}
\label{sec:ddg4-user-manual-geant4action-base}
%=============================================================================

\noindent
The support of interactive in Geant4 is absolutely mandatory to debug detector
setups in small steps. The Geant4 toolkit did provide for this reason 
a machinery of UI commands.
\begin{figure}[h]
  \begin{center}
    \includegraphics[height=70mm] {DDG4-UIMessenger.png}
    \caption{The design of the \tts{Geant4UIMessenger} class responsible for
        the interaction between the user and the components of \DDG and Geant4.}
    \label{fig:ddg4-tracking-action}
  \end{center}
\end{figure}

\noindent
The UI control is enabled, as soon as the property "Control" (boolean) is set to true.
Be default all properties of the action are exported.
Similar to the callback mechanism described above it is also feasible to
register any object callback invoking a method of a \tts{Geant4Action}-subclass. 

\noindent
The following (shortened) screen dump illustrates the usage of the 
generic interface any Geant4Action offers:
\begin{unnumberedcode}
Idle> ls
Command directory path : /
 Sub-directories : 
   /control/   UI control commands.
   /units/   Available units.
   /process/   Process Table control commands.
   /ddg4/   Control for all named Geant4 actions
   ...
Idle> cd /ddg4
Idle> ls
...
Control for all named Geant4 actions

 Sub-directories : 
   /ddg4/EventAction/   Control hierarchy for Geant4 action:EventAction
   /ddg4/RunAction/   Control hierarchy for Geant4 action:RunAction
   /ddg4/Gun/   Control hierarchy for Geant4 action:Gun
   /ddg4/GeneratorAction/   Control hierarchy for Geant4 action:GeneratorAction
   /ddg4/PhysicsList/   Control hierarchy for Geant4 action:PhysicsList
Idle> ls Gun            
...
Control hierarchy for Geant4 action:Gun

 Sub-directories : 
 Commands : 
   show * Show all properties of Geant4 component:Gun
...
   particle * Property item of type std::string
   pos_x * Property item of type double
   pos_y * Property item of type double
   pos_z * Property item of type double
Idle> Gun/show
...
PropertyManager: Property multiplicity = 1
PropertyManager: Property name = 'Gun'
PropertyManager: Property particle = 'e-'
PropertyManager: Property pos_x = 0
PropertyManager: Property pos_y = 0
PropertyManager: Property pos_z = 0

Idle> Gun/pos_z 1.0   
Geant4UIMessenger: +++ Gun> Setting new property value pos_z = 1.0.
Idle> Gun/pos_y 1.0
Geant4UIMessenger: +++ Gun> Setting new property value pos_y = 1.0.
Idle> Gun/pos_x 1.0
Geant4UIMessenger: +++ Gun> Setting new property value pos_x = 1.0.
Idle> Gun/show
...
PropertyManager: Property pos_x = 1
PropertyManager: Property pos_y = 1
PropertyManager: Property pos_z = 1

\end{unnumberedcode}


\newpage
%=============================================================================
\section{Setting up DDG4}
\label{sec:ddg4-implementation-setup}
%=============================================================================

\noindent
\DDG offers several possibilities to configure a simulation application
using
\begin{itemize}\itemcompact
\item XML files,
\item by coding a setup script loaded from the \tts{ROOT} interpreter 
	with the AClick	mechanism.
\item by creating a setup script using \tts{python} and 
	\tts{ROOT}'s reflection mechanism exposed by \tts{PyROOT}.
\end{itemize}
The follwing subsection describe these different mechanism.


%=============================================================================
\subsection{Setting up DDG4 using XML}
\label{sec:ddg4-implementation-setup-xml}
%=============================================================================

\noindent
A special plugin was developed to enable the configuration of \DDG using
XML structures. These files are parsed identically to the geometry setup
in \DDhep the only difference is the name of the root-element, which for 
\DDG is \tts{<geant4\_setup>}. 
The following code snippet shows the basic structure of a \DDG setup file:
\begin{unnumberedcode}
<geant4_setup>
  <physicslist>          ,,,  </physicslist>  <!-- Defintiion of the physics list          -->
  <actions>              ...  </actions>      <!-- The list of global actions              -->
  <phases>               ...  </phases>       <!-- The definition of the various phases    -->
  <filters>              ...  </filters>      <!-- The list of global filter actions       -->
  <sequences>            ...  </sequences>    <!-- The list of defined sequences           -->
  <sensitive_detectors>  ...  </sensitive_detectors>  <!-- The list of sensitive detectors -->
  <properties>           ...  </properties>   <!-- Free format option sequences            -->
</geant4_setup>
\end{unnumberedcode}
To setup a \DDG4 application any number of xml setup files may be interpreted 
iteratively. In the following subsections the content of these first level sub-trees will
be discussed.

%=============================================================================
\subsubsection{Setup of the Physics List}
\label{sec:ddg4-setup-xml-physicslist}
%=============================================================================

\noindent
The main tag to setup a physics list is \tts{<physicslist>} with the 
\tts{name} attribute defining the instance of the \tts{Geant4PhysicsList} object.
An example code snippet is shown below in Figure~\ref{fig:ddg4-setup-xml-physicslist}.

\begin{code}
<geant4_setup>
  <physicslist name="Geant4PhysicsList/MyPhysics.0">

    <extends name="QGSP_BERT"/>                    <!-- Geant4 basic Physics list -->

    <particles>                                    <!-- Particle constructors     -->
      <construct name="G4Geantino"/>
      <construct name="G4ChargedGeantino"/>
      <construct name="G4Electron"/>
      <construct name="G4Gamma"/>
      <construct name="G4BosonConstructor"/>
      <construct name="G4LeptonConstructor"/>
      <construct name="G4MesonConstructor"/>
      <construct name="G4BaryonConstructor"/>
      ...
    </particles>

    <processes>                                    <!-- Process constructors      -->
      <particle name="e[+-]" cut="1*mm">
        <process name="G4eMultipleScattering"  ordAtRestDoIt="-1"       ordAlongSteptDoIt="1"
                                               ordPostStepDoIt="1"/>
        <process name="G4eIonisation"          ordAtRestDoIt="-1"       ordAlongSteptDoIt="2"
                                               ordPostStepDoIt="2"/>
      </particle>
      <particle name="mu[+-]">
        <process name="G4MuMultipleScattering" ordAtRestDoIt="-1"       ordAlongSteptDoIt="1"     
                                               ordPostStepDoIt="1"/>
        <process name="G4MuIonisation"         ordAtRestDoIt="-1"       ordAlongSteptDoIt="2"
                                               ordPostStepDoIt="2"/>
      </particle>
      ...
    </processes>

    <physics>                                      <!-- Physics constructors      -->
      <construct name="G4EmStandardPhysics"/>
      <construct name="HadronPhysicsQGSP"/>
      ...
    </physics>
    
  </physicslist>
</geant4_setup>
\end{code}
\begin{figure}[h]
\caption{XML snippet showing the configuration of a physics list.}
\label{fig:ddg4-setup-xml-physicslist}
\end{figure}


To trigger a call to a
\begin{itemize}\itemcompact
\item \bold{particle constructors} (line 7-14), use the \tts{<particles>} section 
    and define the Geant4 particle constructor to be called by name. To trigger a call to
\item \bold{physics process constructors}, as shown in line 19-30, 
    Define for each particle matching the name pattern (regular expression!) and the 
    default cut value for the corresponding processes. The attributes ordXXXX correspond
    to the arguments of the Geant4 call \\
    \tts{G4ProcessManager::AddProcess(process,ordAtRestDoIt, ordAlongSteptDoIt,ordPostStepDoIt);}
    The processes themself are created using the ROOT plugin mechanism.
    To trigger a call to
\item \bold{physics constructors}, as shown in line 34-35, use the \tts{<physics>} section
    and 
\item to base all these constructs on an already existing predefined Geant4 physics list
    use the \tts{<extends>} tag with the attribute containing the name of the physics list
    as shown in line 4.
\end{itemize}
If only a predefined physics list is used, which probably already satisfies very many use cases,
all these section collapse to:
\begin{code}
<geant4_setup>
  <physicslist name="Geant4PhysicsList/MyPhysics.0">
    <extends name="QGSP_BERT"/>                    <!-- Geant4 basic Physics list -->
  </physicslist>
</geant4_setup>
\end{code}

%=============================================================================
\subsubsection{Setup of Global Geant4 Actions}
\label{sec:ddg4-setup-xml-geant4-actions}
%=============================================================================

\noindent
Global actions must be defined in the \tts{<actions>} section as shown in the following snippet:
\begin{code}
<geant4_setup>
  <actions>
    <action name="Geant4TestRunAction/RunInit">
      <properties Property_int="12345"
          Property_double="-5e15"
          Property_string="Startrun: Hello_2"/>
     </action>
    <action name="Geant4TestEventAction/UserEvent_2"
            Property_int="1234"
            Property_double="5e15"
            Property_string="Hello_2" />
  </actions>
</geant4_setup>
\end{code}
The default properties of \bold{every} \tts{Geant4Action} object are:
\begin{unnumberedcode}
Name        [string]                Action name
OutputLevel [int]                   Flag to customize the level of printout
Control     [boolean]               Flag if the UI messenger should be installed.
\end{unnumberedcode}
The \tts{name} attribute of an action child is a qualified name: The first part
denotes the type of the plugin (i.e. its class), the second part the name of the instance.
Within one collection the instance \tts{name} must be unique.
Properties of Geant4Actions are set by placing them as attributes into the
\tts{<properties>} section.

%=============================================================================
\subsubsection{Setup of Geant4 Filters}
\label{sec:ddg4-setup-xml-geant4-filters}
%=============================================================================
\noindent
Filters are special actions called by \tts{Geant4Sensitive}s. 
Filters may be global or anonymous i.e. reusable by several sensitive detector
sequences as illustrated in Section~\ref{sec:ddg4-setup-xml-geant4-sequences}. 
The setup is analogous to the setup of global actions:
\begin{code}
  <filters>
    <filter name="GeantinoRejectFilter/GeantinoRejector"/>
    <filter name="ParticleRejectFilter/OpticalPhotonRejector">
      <properties particle="opticalphoton"/>
    </filter>
    <filter name="ParticleSelectFilter/OpticalPhotonSelector">
      <properties particle="opticalphoton"/>
    </filter>
    <filter name="EnergyDepositMinimumCut">
      <properties Cut="10*MeV"/>
    </filter>
    <!--        ... next global filter ...       -->
  </filters>
\end{code}
Global filters are accessible from the \tts{Geant4Kernel} object.

%=============================================================================
\subsubsection{Geant4 Action Sequences}
\label{sec:ddg4-setup-xml-geant4-sequences}
%=============================================================================

\noindent
\tts{Geant4 Action Sequences} by definition are \tts{Geant4Action} objects.
Hence, they share the setup mechanism with properties etc. For the setup
mechanism two different types of sequences are known to \DDG:
{\it{Action sequences}} and {\it{Sensitive detector sequences}}. Bot are declared in
the \tts{sequences} section:
\begin{code}
<geant4_setup>
  <sequences>
    <sequence name="Geant4EventActionSequence/EventAction"> <!-- Sequence "EventAction" of type
                                                                 "Geant4EventActionSequence" -->
      <action name="Geant4TestEventAction/UserEvent_1">     <!-- Anonymouns action                   -->
        <properties Property_int="01234"                    <!-- Properties go inline               -->
            Property_double="1e11"
            Property_string="'Hello_1'"/>
      </action>
      <action name="UserEvent_2"/>                          <!-- Global action defined in "actions" -->
                                                            <!-- Only the name is referenced here   -->
      <action name="Geant4Output2ROOT/RootOutput">          <!-- ROOT I/O action                    -->
        <properties Output="simple.root"/>                  <!-- Output file property               -->
      </action>
      <action name="Geant4Output2LCIO/LCIOOutput">          <!-- LCIO output action                 -->
        <properties Output="simple_lcio"/>                  <!-- Output file property               -->
      </action>
    </sequence>


    <sequence sd="SiTrackerBarrel" type="Geant4SensDetActionSequence">
      <filter name="GeantinoRejector"/>
      <filter name="EnergyDepositMinimumCut"/>
      <action name="Geant4SimpleTrackerAction/SiTrackerBarrelHandler"/>
    </sequence>
    <sequence sd="SiTrackerEndcap" type="Geant4SensDetActionSequence">
      <filter name="GeantinoRejector"/>
      <filter name="EnergyDepositMinimumCut"/>
      <action name="Geant4SimpleTrackerAction/SiTrackerEndcapHandler"/>
    </sequence>
    <!--    ... next sequence ...     -->
  </sequences>
</geant4_setup>
\end{code}
Here firstly the \bold{EventAction} sequence is defined with its members. 
Secondly a sensitve detector sequence is defined for the subdetector
\tts{SiTrackerBarrel} of type \tts{Geant4SensDetActionSequence}.
The sequence uses two filters: \tts{GeantinoRejector} to not generate hits
from geantinos and \tts{EnergyDepositMinimumCut} to enforce a minimal energy deposit.
These filters are global i.e. they may be applied by many subdetectors.
The setup of global filters is described in 
Section~\ref{sec:ddg4-setup-xml-geant4-filters}.
Finally the action \tts{SiTrackerEndcapHandler} of type \tts{Geant4SimpleTrackerAction}
is chained, which collects the deposited energy and 
creates a collection of hits. The \tts{Geant4SimpleTrackerAction} is a template
callback to illustrate the usage of sensitive elements in \DDG.
The resulting hit collection of these handlers by default have the same name as the
object instance name.
Analogous below the sensitive detector sequence for the subdetector 
\tts{SiTrackerEndcap} is shown, which reuses the same filter actions, but will build its own
hit collection.

\noindent
\bold{Plase note:} 
\begin{itemize}\itemcompact
\item \bold{It was already mentioned, but once again}: Event-, run-, generator-, tracking-,
    stepping- and stacking actions sequences have predefined names! 
    These names are fixed and part of the \bold{common knowledge}, they cannot be altered.
    Please refer to 
    Section~\ref{sec:ddg4-user-manual-implementation-geant4action-sequences} 
    for the names of the global action sequences.
\item the sensitive detector sequences are matched by the attribute \tts{sd} to the 
    subdetectors created with the \DDhep detector description package. Values must match!
\item In the event that several xml files are parsed it is absolutely vital that 
    the \tts{<actions>} section is interpreted \bold{before} the \tts{sequences}.
\item For each XML file several \tts{<sequences>} are allowed.
\noindent
\end{itemize}

%=============================================================================
\subsubsection{Setup of Geant4 Sensitive Detectors}
\label{sec:ddg4-setup-xml-geant4-sensitive detectors}
%=============================================================================
\begin{code}
  <geant4_setup>
    <sensitive_detectors>
      <sd name="SiTrackerBarrel" 
          type="Geant4SensDet" 
          ecut="10.0*MeV" 
          verbose="true" 
          hit_aggregation="position">
      </sd>
      <!-- ...  next sensitive detector ... -->
    </sensitive_detectors>
  </geant4_setup>
\end{code}



%=============================================================================
\subsubsection{Miscellaneous Setup of Geant4 Objects}
\label{sec:ddg4-setup-xml-geant4-objects}
%=============================================================================

\noindent
This section is used for the flexible setup of auxiliary objects such as the 
electromagnetic fields used in Geant4:
\begin{code}
  <geant4_setup>
    <properties>
      <attributes name="geant4_field"
            id="0"
            type="Geant4FieldSetup"
            object="GlobalSolenoid"
            global="true"
            min_chord_step="0.01*mm"
            delta_chord="0.25*mm"
            delta_intersection="1e-05*mm"
            delta_one_step="0.001*mm"
            eps_min="5e-05*mm"
            eps_max="0.001*mm"
            stepper="HelixSimpleRunge"
            equation="Mag_UsualEqRhs">
      </attributes>
      ...
    </properties>
  </geant4_setup>
\end{code}
Important are the tags \tts{type} and \tts{object}, which are used to firstly
define the plugin to be called and secondly define the object from the \DDhep
description to be configured for the use within Geant4.

%=============================================================================
\subsubsection{Setup of Geant4 Phases}
\label{sec:ddg4-setup-xml-geant4-phases}
%=============================================================================

\noindent
Phases are configured as shown below.
However, the use is \bold{discouraged},
since it is not yet clear if there are appropriate use cases!
\begin{code}
  <phases>
    <phase type="RunAction/begin">
      <action name="RunInit"/>
      <action name="Geant4TestRunAction/UserRunInit">
    <properties Property_int="1234"
            Property_double="5e15"
            Property_string="'Hello_2'"/>
      </action>
    </phase>
    <phase type="EventAction/begin">
      <action name="UserEvent_2"/>
    </phase>
    <phase type="EventAction/end">
      <action name="UserEvent_2"/>
    </phase>
    ...
  </phases>
\end{code}

\newpage
%=============================================================================
\subsection{Setting up DDG4 using ROOT-CINT}
\label{sec:ddg4-implementation-setup-root-cint}
%=============================================================================

\noindent
The setup of \DDG directly from the the ROOT interpreter using the AClick
mechanism is very simple, but mainly meant for purists (like me ;-)),
since it is nearly equivalent to the explicit setup within a \tts{C++} 
main program.
The following code section shows how to do it. For explanation the code
secment is discussed below line by line.
\begin{code}
#include "DDG4/Geant4Config.h"
#include "DDG4/Geant4TestActions.h"
#include "DDG4/Geant4TrackHandler.h"
#include <iostream>

using namespace std;
using namespace DD4hep;
using namespace DD4hep::Simulation;
using namespace DD4hep::Simulation::Test;
using namespace DD4hep::Simulation::Setup;

#if defined(__MAKECINT__)
#pragma link C++ class Geant4RunActionSequence;
#pragma link C++ class Geant4EventActionSequence;
#pragma link C++ class Geant4SteppingActionSequence;
#pragma link C++ class Geant4StackingActionSequence;
#pragma link C++ class Geant4GeneratorActionSequence;
#pragma link C++ class Geant4Action;
#pragma link C++ class Geant4Kernel;
#endif

SensitiveSeq::handled_type* setupDetector(Kernel& kernel, const std::string& name)   {
  SensitiveSeq sd = SensitiveSeq(kernel,name);
  Sensitive  sens = Sensitive(kernel,"Geant4TestSensitive/"+name+"Handler",name);
  sd->adopt(sens);
  sens = Sensitive(kernel,"Geant4TestSensitive/"+name+"Monitor",name);
  sd->adopt(sens);
  return sd;
}

void exampleAClick()  {
  Geant4Kernel& kernel = Geant4Kernel::instance(LCDD::getInstance());
  kernel.loadGeometry("file:../DD4hep.trunk/DDExamples/CLICSiD/compact/compact.xml");
  kernel.loadXML("DDG4_field.xml");

  GenAction gun(kernel,"Geant4ParticleGun/Gun");
  gun["energy"] = 0.5*GeV;                          // Set properties
  gun["particle"] = "e-";
  gun["multiplicity"] = 1;
  kernel.generatorAction().adopt(gun);

  Action run_init(kernel,"Geant4TestRunAction/RunInit");
  run_init["Property_int"] = 12345;
  kernel.runAction().callAtBegin  (run_init.get(),&Geant4TestRunAction::begin);
  kernel.eventAction().callAtBegin(run_init.get(),&Geant4TestRunAction::beginEvent);
  kernel.eventAction().callAtEnd  (run_init.get(),&Geant4TestRunAction::endEvent);

  Action evt_1(kernel,"Geant4TestEventAction/UserEvent_1");
  evt_1["Property_int"] = 12345;                    // Set properties
  evt_1["Property_string"] = "Events";
  kernel.eventAction().adopt(evt_1);

  EventAction evt_2(kernel,"Geant4TestEventAction/UserEvent_2");
  kernel.eventAction().adopt(evt_2);

  kernel.runAction().callAtBegin(evt_2.get(),&Geant4TestEventAction::begin);
  kernel.runAction().callAtEnd  (evt_2.get(),&Geant4TestEventAction::end);
 
  setupDetector(kernel,"SiVertexBarrel");
  setupDetector(kernel,"SiVertexEndcap");
  // .... more subdetectors here .....
  setupDetector(kernel,"LumiCal");
  setupDetector(kernel,"BeamCal");

  kernel.configure();
  kernel.initialize();
  kernel.run();
  std::cout << "Successfully executed application .... " << std::endl;
  kernel.terminate();
}
\end{code}

\noindent
\begin{tabular} {l||p{0cm}}
\docline{Line}{}
\docline{1}{The header file \tts{Geant4Config.h} contains a set of wrapper
    classes to easy the creation of objects using the plugin mechanism and setting
    properties to \tts{Geant4Action} objects. These helpers and the corresponding 
    functionality are not included in the wrapped classes themselves to not 
    clutter the code with stuff only used for the setup.
    All contained objects are in the namespace \tts{DD4hep::Simulation::Setup}}.
\docline{6-10}{Save yourself specifying all the namespaces objects are in....}
\docline{13-19}{CINT processing pragmas. 
    Classes defined here will be available at the ROOT prompt
    after this AClick is loaded.}
\docline{22-29}{Sampler to fill the sensitive detector sequences for each 
    subdetector with two entries: a handler and a monitor action.
    Please note, that this here is example code and in real life specialized actions
    will have to be provided for each subdetector.}
\docline{31}{Let's go for it. here the entry point starts....}
\docline{32}{Create the \tts{Geant4Kernel} object.}
\docline{33}{Load the geometry into \DDhep.}
\docline{34}{Redefine the setup of the sensitive detectors.}
\docline{36-40}{Create the generator action of type \tts{Geant4ParticleGun} with name
    \tts{Gun}, set non-default properties and activate the configured object
    by attaching it to the \tts{Geant4Kernel}.}
\docline{42-46}{Create a user defined begin-of-run action callback, set the properties
    and attach it to the begin of run calls. To collect statistics extra member functions
    are registered to be called at the beginning and the end of each event.}
\docline{48-51}{Create a user defined event action routine, set its properties
    and attach it to the event action sequence.}
\docline{53-54}{Create a second event action and register it to the event action sequence.
    This action will be called after the previously created action.}
\docline{56-57}{For this event action we want to receive callbacks at start- 
    and end-of-run to produce additional summary output.}
\docline{59-63}{Call the sampler routine to attach test actions to the subdetectors defined.}
\docline{65-66}{Configure, initialize and run the Geant4 application.
    Most of the Geant4 actions will only be created here and the action sequences
    created before will be attached now.}
\docline{69}{Terminate the Geant4 application and exit.}
\end{tabular}

\newpage
\noindent
CINT currently cannot handle pointers to member functions~\footnote{This may change
in the future once ROOT uses \tts{clang} and \tts{cling} as the interpreting engine.}. 
Hence the above AClick only works in compiled mode. To invoke the compilation the following 
action is necessary from the ROOT prompt:


\begin{code}
$> root.exe
  *******************************************
  *                                         *
  *        W E L C O M E  to  R O O T       *
  *                                         *
  *   Version   5.34/10    29 August 2013   *
  *                                         *
  *  You are welcome to visit our Web site  *
  *          http://root.cern.ch            *
  *                                         *
  *******************************************

ROOT 5.34/10 (heads/v5-34-00-patches@v5-34-10-5-g0e8bac8, Sep 04 2013, 11:52:19 on linux)

CINT/ROOT C/C++ Interpreter version 5.18.00, July 2, 2010
Type ? for help. Commands must be C++ statements.
Enclose multiple statements between { }.
root [0] .X initAClick.C
.... Setting up the CINT include pathes and the link statements.

root [1] .L ../DD4hep.trunk/DDG4/examples/exampleAClick.C+
Info in <TUnixSystem::ACLiC>: creating shared library ....exampleAClick_C.so
.... some Cint warnings concerning member function pointers .....

root [2] exampleAClick()
.... and it starts ...
\end{code}

\noindent
The above scripts are present in the DDG4/example directory located in svn.
The intialization script \tts{initAClick.C} may require customization
to cope with the installation pathes.

%=============================================================================
\subsection{Setting up DDG4 using Python}
\label{sec:ddg4-implementation-setup-python}
%=============================================================================

\noindent
Given the reflection interface of ROOT, the setup of the simulation interface
using DD4hep is of course also possible using the python interpreted language.
In the following code example the setup of Geant4 using the \tw{ClicSid} 
example is shown using python~\footnote{For comparison, the same example was 
used to illustrate the setup using XML files.}.

\begin{code}
import DDG4
from SystemOfUnits import *

"""

   DD4hep example setup using the python configuration

   @author  M.Frank
   @version 1.0

"""
def run():
  kernel = DDG4.Kernel()
  kernel.loadGeometry("file:../DD4hep.trunk/DDExamples/CLICSiD/compact/compact.xml")
  kernel.loadXML("DDG4_field.xml")

  lcdd = kernel.lcdd()
  print '+++   List of sensitive detectors:'
  for i in lcdd.detectors(): 
    o = DDG4.DetElement(i.second)
    sd = lcdd.sensitiveDetector(o.name())
    if sd.isValid():
      print '+++  %-32s type:%s'%(o.name(), sd.type(), )

  # Configure Run actions
  run1 = DDG4.RunAction(kernel,'Geant4TestRunAction/RunInit')
  run1.Property_int    = 12345
  run1.Property_double = -5e15*keV
  run1.Property_string = 'Startrun: Hello_2'
  print run1.Property_string, run1.Property_double, run1.Property_int
  run1.enableUI()
  kernel.registerGlobalAction(run1)
  kernel.runAction().add(run1)

  # Configure Event actions
  evt2 = DDG4.EventAction(kernel,'Geant4TestEventAction/UserEvent_2')
  evt2.Property_int    = 123454321
  evt2.Property_double = 5e15*GeV
  evt2.Property_string = 'Hello_2 from the python setup'
  evt2.enableUI()
  kernel.registerGlobalAction(evt2)

  evt1 = DDG4.EventAction(kernel,'Geant4TestEventAction/UserEvent_1')
  evt1.Property_int=01234
  evt1.Property_double=1e11
  evt1.Property_string='Hello_1'
  evt1.enableUI()

  kernel.eventAction().add(evt1)
  kernel.eventAction().add(evt2)

  # Configure I/O
  evt_root = DDG4.EventAction(kernel,'Geant4Output2ROOT/RootOutput')
  evt_root.Control = True
  evt_root.Output = "simple.root"
  evt_root.enableUI()

  evt_lcio = DDG4.EventAction(kernel,'Geant4Output2LCIO/LcioOutput')
  evt_lcio.Output = "simple_lcio"
  evt_lcio.enableUI()

  kernel.eventAction().add(evt_root)
  kernel.eventAction().add(evt_lcio)

  # Setup particle gun
  gun = DDG4.GeneratorAction(kernel,"Geant4ParticleGun/Gun")
  gun.energy   = 0.5*GeV
  gun.particle = 'e-'
  gun.multiplicity = 1
  gun.enableUI()
  kernel.generatorAction().add(gun)

  # Setup global filters fur use in sensntive detectors
  f1 = DDG4.Filter(kernel,'GeantinoRejectFilter/GeantinoRejector')
  f2 = DDG4.Filter(kernel,'ParticleRejectFilter/OpticalPhotonRejector')
  f2.particle = 'opticalphoton'
  f3 = DDG4.Filter(kernel,'ParticleSelectFilter/OpticalPhotonSelector') 
  f3.particle = 'opticalphoton'
  f4 = DDG4.Filter(kernel,'EnergyDepositMinimumCut')
  f4.Cut = 10*MeV
  f4.enableUI()
  kernel.registerGlobalFilter(f1)
  kernel.registerGlobalFilter(f2)
  kernel.registerGlobalFilter(f3)
  kernel.registerGlobalFilter(f4)

  # First the tracking detectors
  seq = DDG4.SensitiveSequence(kernel,'Geant4SensDetActionSequence/SiVertexBarrel')
  act = DDG4.SensitiveAction(kernel,'Geant4SimpleTrackerAction/SiVertexBarrelHandler','SiVertexBarrel')
  seq.add(act)
  seq.add(f1)
  seq.add(f4)
  act.add(f1)

  seq = DDG4.SensitiveSequence(kernel,'Geant4SensDetActionSequence/SiVertexEndcap')
  act = DDG4.SensitiveAction(kernel,'Geant4SimpleTrackerAction/SiVertexEndcapHandler','SiVertexEndcap')
  seq.add(act)
  seq.add(f1)
  seq.add(f4)

  seq = DDG4.SensitiveSequence(kernel,'Geant4SensDetActionSequence/SiTrackerBarrel')
  act = DDG4.SensitiveAction(kernel,'Geant4SimpleTrackerAction/SiTrackerBarrelHandler','SiTrackerBarrel')
  seq.add(act)
  seq.add(f1)
  seq.add(f4)

  seq = DDG4.SensitiveSequence(kernel,'Geant4SensDetActionSequence/SiTrackerEndcap')
  act = DDG4.SensitiveAction(kernel,'Geant4SimpleTrackerAction/SiTrackerEndcapHandler','SiTrackerEndcap')
  seq.add(act)

  seq = DDG4.SensitiveSequence(kernel,'Geant4SensDetActionSequence/SiTrackerForward')
  act = DDG4.SensitiveAction(kernel,'Geant4SimpleTrackerAction/SiTrackerForwardHandler','SiTrackerForward')
  seq.add(act)

  # Now the calorimeters
  seq = DDG4.SensitiveSequence(kernel,'Geant4SensDetActionSequence/EcalBarrel')
  act = DDG4.SensitiveAction(kernel,'Geant4SimpleCalorimeterAction/EcalBarrelHandler','EcalBarrel')
  seq.add(act)

  seq = DDG4.SensitiveSequence(kernel,'Geant4SensDetActionSequence/EcalEndcap')
  act = DDG4.SensitiveAction(kernel,'Geant4SimpleCalorimeterAction/EcalEndCapHandler','EcalEndcap')
  seq.add(act)

  seq = DDG4.SensitiveSequence(kernel,'Geant4SensDetActionSequence/HcalBarrel')
  act = DDG4.SensitiveAction(kernel,'Geant4SimpleCalorimeterAction/HcalBarrelHandler','HcalBarrel')
  act.adoptFilter(kernel.globalFilter('OpticalPhotonRejector'))
  seq.add(act)

  act = DDG4.SensitiveAction(kernel,'Geant4SimpleCalorimeterAction/HcalOpticalBarrelHandler','HcalBarrel')
  act.adoptFilter(kernel.globalFilter('OpticalPhotonSelector'))
  seq.add(act)

  seq = DDG4.SensitiveSequence(kernel,'Geant4SensDetActionSequence/HcalEndcap')
  act = DDG4.SensitiveAction(kernel,'Geant4SimpleCalorimeterAction/HcalEndcapHandler','HcalEndcap')
  seq.add(act)

  seq = DDG4.SensitiveSequence(kernel,'Geant4SensDetActionSequence/HcalPlug')
  act = DDG4.SensitiveAction(kernel,'Geant4SimpleCalorimeterAction/HcalPlugHandler','HcalPlug')
  seq.add(act)

  seq = DDG4.SensitiveSequence(kernel,'Geant4SensDetActionSequence/MuonBarrel')
  act = DDG4.SensitiveAction(kernel,'Geant4SimpleCalorimeterAction/MuonBarrelHandler','MuonBarrel')
  seq.add(act)

  seq = DDG4.SensitiveSequence(kernel,'Geant4SensDetActionSequence/MuonEndcap')
  act = DDG4.SensitiveAction(kernel,'Geant4SimpleCalorimeterAction/MuonEndcapHandler','MuonEndcap')
  seq.add(act)

  seq = DDG4.SensitiveSequence(kernel,'Geant4SensDetActionSequence/LumiCal')
  act = DDG4.SensitiveAction(kernel,'Geant4SimpleCalorimeterAction/LumiCalHandler','LumiCal')
  seq.add(act)

  seq = DDG4.SensitiveSequence(kernel,'Geant4SensDetActionSequence/BeamCal')
  act = DDG4.SensitiveAction(kernel,'Geant4SimpleCalorimeterAction/BeamCalHandler','BeamCal')
  seq.add(act)

  # Now build the physics list:
  phys = kernel.physicsList()
  phys.extends = 'FTFP_BERT'
  #phys.transportation = True
  phys.decays  = True
  phys.enableUI()

  ph = DDG4.PhysicsList(kernel,'Geant4PhysicsList/Myphysics')
  ph.addParticleConstructor('G4BosonConstructor')
  ph.addParticleConstructor('G4LeptonConstructor')
  ph.addParticleProcess('e[+-]','G4eMultipleScattering',-1,1,1)
  ph.addPhysicsConstructor('G4OpticalPhysics')
  ph.enableUI()
  phys.add(ph)

  phys.dump()

  kernel.configure()
  kernel.initialize()
  kernel.run()
  kernel.terminate()

if __name__ == "__main__":
  run()

\end{code}

\newpage
%=============================================================================
\section{Existing DDG4 components}
%=============================================================================
\noindent
In the introduction the longterm goal was expressed, that with DDG4 users
should be able to pick components from a growing palette and connect the
selected components using the setup mechanisms described in 
Section~\ref{sec:ddg4-implementation-setup}.

\noindent
Such a palette based approach obviously depends on the availibility of
documentation for exisisting components describing the properties
of each component and the interaction of each component within the \DDG
framework.

\noindent
All components defer from the basic type \tts{Geant4Action}. This means 
\bold{all} components have the \bold{default} properties described in the
table below:

\vspace{0.5cm}
\begin{tabular}{ l l p{9cm} }
\hline
Component Properties: &  & \tts{default} \\
\hline
\bold{OuputLevel}     & [int]  & Output level of the component to customize printouts             \\
\bold{Name}           & [string]  & Component name [read-only] \\
\bold{Control}        & [boolean] & Steering of the Geant4 Messenger creation \\
\hline
\end{tabular}


\vspace{5cm}

\begin{center}
{\large{\bf{
\begin{tabular} {| p{15cm} |}
\hline\space  \\

\noindent
{\underline{Important notice for developers:}} \\

\noindent
Since the documentation of developed components is VERY important,
please never forget to supply the corresponding documentation.\\
\\
\noindent
At least supply the minimal documentation ash shown below
in the appended examples for the "Simple" detector response and I/O
components.
\\ \space\hline 
\end{tabular}
}}}
\end{center}
\clearpage


%=============================================================================
\subsection{Physics Lists}
%=============================================================================
\noindent


%=============================================================================
\subsection{Sensitive Detectors}
%=============================================================================
\noindent

%=============================================================================
\subsubsection{Geant4SimpleTracker}
%=============================================================================
\noindent
Simple sensitive detector for tracking detectors. These trackers create one
single hit collection. The created hits may be written out with the output
modules of type "Simple" described in 
Section~\ref{sec:ddg4-components-IO-ROOT-simple} 
and~\ref{sec:ddg4-components-IO-ROOT-simple}. Here the basic specifications:

\vspace{0.5cm}
\begin{tabular}{ l p{10cm} }
\hline
Basics: & \\
\hline
\bold{Class name}      & \tts{Geant4SensitiveAction<SimpleTracker>}  \\
\bold{File name}       & \tts{DDG4/plugins/Geant4SDActions.cpp}      \\
\bold{Hit collection}  & \tts{<detector name>}                       \\
\bold{Hit class}       & \tts{SimpleTracker::Hit}                    \\
\bold{File name}       & \tts{DDG4/include/Geant4Data.h}             \\
\hline
Component Properties:  & \tts{default} \\
\hline
\end{tabular}

%=============================================================================
\subsubsection{Geant4SimpleCalorimeter}
%=============================================================================
\noindent
Simple sensitive detector for calorimeters. The sensitive detector creates one
single hit collection. The created hits may be written out with the output
modules of type "Simple" described in 
Section~\ref{sec:ddg4-components-IO-ROOT-simple} 
and~\ref{sec:ddg4-components-IO-ROOT-simple}. Here the basic specifications:

\vspace{0.5cm}
\begin{tabular}{ l p{10cm} }
\hline
Basics: & \\
\hline
\bold{Class name}      & \tts{Geant4SensitiveAction<SimpleCalorimeter>}  \\
\bold{File name}       & \tts{DDG4/plugins/Geant4SDActions.cpp}      \\
\bold{Hit collection}  & \tts{<detector name>}                       \\
\bold{Hit class}       & \tts{SimpleCalorimeter::Hit}                \\
\bold{File name}       & \tts{DDG4/include/Geant4Data.h}             \\
\hline
Component Properties:  & \tts{default} \\
\hline
\end{tabular}


%=============================================================================
\subsubsection{Geant4SimpleOpticalCalorimeter}
%=============================================================================
\noindent


%=============================================================================
\subsection{I/O Components}
%=============================================================================
\noindent

%=============================================================================
\subsubsection{ROOT Output "Simple"}
\label{sec:ddg4-components-IO-ROOT-simple}
%=============================================================================
\noindent

%=============================================================================
\subsubsection{LCIO Output "Simple"}
\label{sec:ddg4-components-IO-LCIO-simple}
%=============================================================================
\noindent



%=============================================================================
\newpage
\begin{thebibliography}{9}
\input{DD4hep-bibliography.tex}
\bibitem{bib:DDSegmentations} C.Grefe et al.,
                   "The DDSegmentation package", 
                   Non existing documentation to be written.
\end{thebibliography}
%=============================================================================
\end{document}
