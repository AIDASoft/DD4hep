
%=============================================================================
\section{Setting up DDG4}
\label{sec:ddg4-implementation-setup}
%=============================================================================

\noindent
\DDG offers several possibilities to configure a simulation application
using
\begin{itemize}\itemcompact
\item XML files,
\item by coding a setup script loaded from the \tts{ROOT} interpreter 
	with the AClick	mechanism.
\item by creating a setup script using \tts{python} and 
	\tts{ROOT}'s reflection mechanism exposed by \tts{PyROOT}.
\end{itemize}
The following subsection describe these different mechanism. An attempt was made
to match the naming conventions of all approaches where possible.

%=============================================================================
\subsection{Setting up DDG4 using XML}
\label{sec:ddg4-implementation-setup-xml}
%=============================================================================

\noindent
A special plugin was developed to enable the configuration of \DDG using
XML structures. These files are parsed identically to the geometry setup
in \DDhep the only difference is the name of the root-element, which for 
\DDG is \tts{<geant4\_setup>}. 
The following code snippet shows the basic structure of a \DDG setup file:
\begin{unnumberedcode}
<geant4_setup>
  <physicslist>          ,,,  </physicslist>  <!-- Definition of the physics list          -->
  <actions>              ...  </actions>      <!-- The list of global actions              -->
  <phases>               ...  </phases>       <!-- The definition of the various phases    -->
  <filters>              ...  </filters>      <!-- The list of global filter actions       -->
  <sequences>            ...  </sequences>    <!-- The list of defined sequences           -->
  <sensitive_detectors>  ...  </sensitive_detectors>  <!-- The list of sensitive detectors -->
  <properties>           ...  </properties>   <!-- Free format option sequences            -->
</geant4_setup>
\end{unnumberedcode}
To setup a \DDG4 application any number of xml setup files may be interpreted 
iteratively. In the following subsections the content of these first level sub-trees will
be discussed.

%=============================================================================
\subsubsection{Setup of the Physics List}
\label{sec:ddg4-setup-xml-physicslist}
%=============================================================================

\noindent
The main tag to setup a physics list is \tts{<physicslist>} with the 
\tts{name} attribute defining the instance of the \tts{Geant4PhysicsList} object.
An example code snippet is shown below in Figure~\ref{fig:ddg4-setup-xml-physicslist}.

\begin{code}
<geant4_setup>
  <physicslist name="Geant4PhysicsList/MyPhysics.0">

    <extends name="QGSP_BERT"/>                    <!-- Geant4 basic Physics list -->

    <particles>                                    <!-- Particle constructors     -->
      <construct name="G4Geantino"/>
      <construct name="G4ChargedGeantino"/>
      <construct name="G4Electron"/>
      <construct name="G4Gamma"/>
      <construct name="G4BosonConstructor"/>
      <construct name="G4LeptonConstructor"/>
      <construct name="G4MesonConstructor"/>
      <construct name="G4BaryonConstructor"/>
      ...
    </particles>

    <processes>                                    <!-- Process constructors      -->
      <particle name="e[+-]" cut="1*mm">
        <process name="G4eMultipleScattering"  ordAtRestDoIt="-1"       ordAlongSteptDoIt="1"
                                               ordPostStepDoIt="1"/>
        <process name="G4eIonisation"          ordAtRestDoIt="-1"       ordAlongSteptDoIt="2"
                                               ordPostStepDoIt="2"/>
      </particle>
      <particle name="mu[+-]">
        <process name="G4MuMultipleScattering" ordAtRestDoIt="-1"       ordAlongSteptDoIt="1"     
                                               ordPostStepDoIt="1"/>
        <process name="G4MuIonisation"         ordAtRestDoIt="-1"       ordAlongSteptDoIt="2"
                                               ordPostStepDoIt="2"/>
      </particle>
      ...
    </processes>

    <physics>                                      <!-- Physics constructors      -->
      <construct name="G4EmStandardPhysics"/>
      <construct name="HadronPhysicsQGSP"/>
      ...
    </physics>
    
  </physicslist>
</geant4_setup>
\end{code}
\begin{figure}[h]
\caption{XML snippet showing the configuration of a physics list.}
\label{fig:ddg4-setup-xml-physicslist}
\end{figure}

\begin{itemize}\itemcompact
\item To base all these constructs on an already existing predefined Geant4 physics list
    use the \tts{<extends>} tag with the attribute containing the name of the physics list
    as shown in line 4.
\item To trigger a call to a \bold{particle constructors} (line 7-14), use the \tts{<particles>} section 
    and define the Geant4 particle constructor to be called by name. To trigger a call to
\item \bold{physics process constructors}, as shown in line 19-30, 
    Define for each particle matching the name pattern (regular expression!) and the 
    default cut value for the corresponding processes. The attributes ordXXXX correspond
    to the arguments of the Geant4 call \\
    \tts{G4ProcessManager::AddProcess(process,ordAtRestDoIt, ordAlongSteptDoIt,ordPostStepDoIt);}
    The processes themself are created using the ROOT plugin mechanism.
    To trigger a call to
\item \bold{physics constructors}, as shown in line 34-35, use the \tts{<physics>} section.
\end{itemize}
If only a predefined physics list is used, which probably already satisfies very many use cases,
all these section collapse to:
\begin{code}
<geant4_setup>
  <physicslist name="Geant4PhysicsList/MyPhysics.0">
    <extends name="QGSP_BERT"/>                    <!-- Geant4 basic Physics list -->
  </physicslist>
</geant4_setup>
\end{code}

%=============================================================================
\subsubsection{Setup of Global Geant4 Actions}
\label{sec:ddg4-setup-xml-geant4-actions}
%=============================================================================

\noindent
Global actions must be defined in the \tts{<actions>} section as shown in the following snippet:
\begin{code}
<geant4_setup>
  <actions>
    <action name="Geant4TestRunAction/RunInit">
      <properties Property_int="12345"
          Property_double="-5e15"
          Property_string="Startrun: Hello_2"/>
     </action>
    <action name="Geant4TestEventAction/UserEvent_2"
            Property_int="1234"
            Property_double="5e15"
            Property_string="Hello_2" />
  </actions>
</geant4_setup>
\end{code}
The default properties of \bold{every} \tts{Geant4Action} object are:
\begin{unnumberedcode}
Name        [string]                Action name
OutputLevel [int]                   Flag to customize the level of printout
Control     [boolean]               Flag if the UI messenger should be installed.
\end{unnumberedcode}
The \tts{name} attribute of an action child is a qualified name: The first part
denotes the type of the plugin (i.e. its class), the second part the name of the instance.
Within one collection the instance \tts{name} must be unique.
Properties of Geant4Actions are set by placing them as attributes into the
\tts{<properties>} section.

%=============================================================================
\subsubsection{Setup of Geant4 Filters}
\label{sec:ddg4-setup-xml-geant4-filters}
%=============================================================================
\noindent
Filters are special actions called by \tts{Geant4Sensitive}s. 
Filters may be global or anonymous i.e. reusable by several sensitive detector
sequences as illustrated in Section~\ref{sec:ddg4-setup-xml-geant4-sequences}. 
The setup is analogous to the setup of global actions:
\begin{code}
  <filters>
    <filter name="GeantinoRejectFilter/GeantinoRejector"/>
    <filter name="ParticleRejectFilter/OpticalPhotonRejector">
      <properties particle="opticalphoton"/>
    </filter>
    <filter name="ParticleSelectFilter/OpticalPhotonSelector">
      <properties particle="opticalphoton"/>
    </filter>
    <filter name="EnergyDepositMinimumCut">
      <properties Cut="10*MeV"/>
    </filter>
    <!--        ... next global filter ...       -->
  </filters>
\end{code}
Global filters are accessible from the \tts{Geant4Kernel} object.

%=============================================================================
\subsubsection{Geant4 Action Sequences}
\label{sec:ddg4-setup-xml-geant4-sequences}
%=============================================================================

\noindent
\tts{Geant4 Action Sequences} by definition are \tts{Geant4Action} objects.
Hence, they share the setup mechanism with properties etc. For the setup
mechanism two different types of sequences are known to \DDG:
{\it{Action sequences}} and {\it{Sensitive detector sequences}}. Bot are declared in
the \tts{sequences} section:
\begin{code}
<geant4_setup>
  <sequences>
    <sequence name="Geant4EventActionSequence/EventAction"> <!-- Sequence "EventAction" of type
                                                                 "Geant4EventActionSequence" -->
      <action name="Geant4TestEventAction/UserEvent_1">     <!-- Anonymous action                   -->
        <properties Property_int="01234"                    <!-- Properties go inline               -->
            Property_double="1e11"
            Property_string="'Hello_1'"/>
      </action>
      <action name="UserEvent_2"/>                          <!-- Global action defined in "actions" -->
                                                            <!-- Only the name is referenced here   -->
      <action name="Geant4Output2ROOT/RootOutput">          <!-- ROOT I/O action                    -->
        <properties Output="simple.root"/>                  <!-- Output file property               -->
      </action>
      <action name="Geant4Output2LCIO/LCIOOutput">          <!-- LCIO output action                 -->
        <properties Output="simple.lcio"/>                  <!-- Output file property               -->
      </action>
    </sequence>


    <sequence sd="SiTrackerBarrel" type="Geant4SensDetActionSequence">
      <filter name="GeantinoRejector"/>
      <filter name="EnergyDepositMinimumCut"/>
      <action name="Geant4SimpleTrackerAction/SiTrackerBarrelHandler"/>
    </sequence>
    <sequence sd="SiTrackerEndcap" type="Geant4SensDetActionSequence">
      <filter name="GeantinoRejector"/>
      <filter name="EnergyDepositMinimumCut"/>
      <action name="Geant4SimpleTrackerAction/SiTrackerEndcapHandler"/>
    </sequence>
    <!--    ... next sequence ...     -->
  </sequences>
</geant4_setup>
\end{code}
Here firstly the \bold{EventAction} sequence is defined with its members. 
Secondly a sensitive detector sequence is defined for the subdetector
\tts{SiTrackerBarrel} of type \tts{Geant4SensDetActionSequence}.
The sequence uses two filters: \tts{GeantinoRejector} to not generate hits
from geantinos and \tts{EnergyDepositMinimumCut} to enforce a minimal energy deposit.
These filters are global i.e. they may be applied by many subdetectors.
The setup of global filters is described in 
Section~\ref{sec:ddg4-setup-xml-geant4-filters}.
Finally the action \tts{SiTrackerEndcapHandler} of type \tts{Geant4SimpleTrackerAction}
is chained, which collects the deposited energy and 
creates a collection of hits. The \tts{Geant4SimpleTrackerAction} is a template
callback to illustrate the usage of sensitive elements in \DDG.
The resulting hit collection of these handlers by default have the same name as the
object instance name.
Analogous below the sensitive detector sequence for the subdetector 
\tts{SiTrackerEndcap} is shown, which reuses the same filter actions, but will build its own
hit collection.

\noindent
\bold{Please note:}
\begin{itemize}\itemcompact
\item \bold{It was already mentioned, but once again}: Event-, run-, generator-, tracking-,
    stepping- and stacking actions sequences have predefined names! 
    These names are fixed and part of the \bold{common knowledge}, they cannot be altered.
    Please refer to 
    Section~\ref{sec:ddg4-user-manual-implementation-geant4action-sequences} 
    for the names of the global action sequences.
\item the sensitive detector sequences are matched by the attribute \tts{sd} to the 
    subdetectors created with the \DDhep detector description package. Values must match!
\item In the event that several xml files are parsed it is absolutely vital that 
    the \tts{<actions>} section is interpreted \bold{before} the \tts{sequences}.
\item For each XML file several \tts{<sequences>} are allowed.
\noindent
\end{itemize}

%=============================================================================
\subsubsection{Setup of Geant4 Sensitive Detectors}
\label{sec:ddg4-setup-xml-geant4-sensitive detectors}
%=============================================================================
\begin{code}
  <geant4_setup>
    <sensitive_detectors>
      <sd name="SiTrackerBarrel" 
          type="Geant4SensDet" 
          ecut="10.0*MeV" 
          verbose="true" 
          hit_aggregation="position">
      </sd>
      <!-- ...  next sensitive detector ... -->
    </sensitive_detectors>
  </geant4_setup>
\end{code}



%=============================================================================
\subsubsection{Miscellaneous Setup of Geant4 Objects}
\label{sec:ddg4-setup-xml-geant4-objects}
%=============================================================================

\noindent
This section is used for the flexible setup of auxiliary objects such as the 
electromagnetic fields used in Geant4:
\begin{code}
  <geant4_setup>
    <properties>
      <attributes name="geant4_field"
            id="0"
            type="Geant4FieldSetup"
            object="GlobalSolenoid"
            global="true"
            min_chord_step="0.01*mm"
            delta_chord="0.25*mm"
            delta_intersection="1e-05*mm"
            delta_one_step="0.001*mm"
            eps_min="5e-05*mm"
            eps_max="0.001*mm"
            largest_step = "10*m"
            stepper="HelixSimpleRunge"
            equation="Mag_UsualEqRhs">
      </attributes>
      ...
    </properties>
  </geant4_setup>
\end{code}
Important are the tags \tts{type} and \tts{object}, which are used to firstly
define the plugin to be called and secondly define the object from the \DDhep
description to be configured for the use within Geant4.

%=============================================================================
\subsubsection{Setup of Geant4 Phases}
\label{sec:ddg4-setup-xml-geant4-phases}
%=============================================================================

\noindent
Phases are configured as shown below.
However, the use is \bold{discouraged},
since it is not yet clear if there are appropriate use cases!
\begin{code}
  <phases>
    <phase type="RunAction/begin">
      <action name="RunInit"/>
      <action name="Geant4TestRunAction/UserRunInit">
    <properties Property_int="1234"
            Property_double="5e15"
            Property_string="'Hello_2'"/>
      </action>
    </phase>
    <phase type="EventAction/begin">
      <action name="UserEvent_2"/>
    </phase>
    <phase type="EventAction/end">
      <action name="UserEvent_2"/>
    </phase>
    ...
  </phases>
\end{code}

\newpage
%=============================================================================
\subsection{Setting up DDG4 using ROOT-CINT}
\label{sec:ddg4-implementation-setup-root-cint}
%=============================================================================

\noindent
The setup of \DDG directly from the the ROOT interpreter using the AClick
mechanism is very simple, but mainly meant for purists (like me ;-)),
since it is nearly equivalent to the explicit setup within a \tts{C++} 
main program.
The following code section shows how to do it. For explanation the code
segment is discussed below line by line.
\begin{code}
#include "DDG4/Geant4Config.h"
#include "DDG4/Geant4TestActions.h"
#include "DDG4/Geant4TrackHandler.h"
#include <iostream>

using namespace std;
using namespace dd4hep;
using namespace dd4hep::sim;
using namespace dd4hep::sim::Test;
using namespace dd4hep::sim::Setup;

#if defined(__MAKECINT__)
#pragma link C++ class Geant4RunActionSequence;
#pragma link C++ class Geant4EventActionSequence;
#pragma link C++ class Geant4SteppingActionSequence;
#pragma link C++ class Geant4StackingActionSequence;
#pragma link C++ class Geant4GeneratorActionSequence;
#pragma link C++ class Geant4Action;
#pragma link C++ class Geant4Kernel;
#endif

SensitiveSeq::handled_type* setupDetector(Kernel& kernel, const std::string& name)   {
  SensitiveSeq sd = SensitiveSeq(kernel,name);
  Sensitive  sens = Sensitive(kernel,"Geant4TestSensitive/"+name+"Handler",name);
  sd->adopt(sens);
  sens = Sensitive(kernel,"Geant4TestSensitive/"+name+"Monitor",name);
  sd->adopt(sens);
  return sd;
}

void exampleAClick()  {
  Geant4Kernel& kernel = Geant4Kernel::instance(Detector::getInstance());
  kernel.loadGeometry("file:../DD4hep.trunk/DDExamples/CLICSiD/compact/compact.xml");
  kernel.loadXML("DDG4_field.xml");

  GenAction gun(kernel,"Geant4ParticleGun/Gun");
  gun["energy"] = 0.5*GeV;                          // Set properties
  gun["particle"] = "e-";
  gun["multiplicity"] = 1;
  kernel.generatorAction().adopt(gun);

  Action run_init(kernel,"Geant4TestRunAction/RunInit");
  run_init["Property_int"] = 12345;
  kernel.runAction().callAtBegin  (run_init.get(),&Geant4TestRunAction::begin);
  kernel.eventAction().callAtBegin(run_init.get(),&Geant4TestRunAction::beginEvent);
  kernel.eventAction().callAtEnd  (run_init.get(),&Geant4TestRunAction::endEvent);

  Action evt_1(kernel,"Geant4TestEventAction/UserEvent_1");
  evt_1["Property_int"] = 12345;                    // Set properties
  evt_1["Property_string"] = "Events";
  kernel.eventAction().adopt(evt_1);

  EventAction evt_2(kernel,"Geant4TestEventAction/UserEvent_2");
  kernel.eventAction().adopt(evt_2);

  kernel.runAction().callAtBegin(evt_2.get(),&Geant4TestEventAction::begin);
  kernel.runAction().callAtEnd  (evt_2.get(),&Geant4TestEventAction::end);
 
  setupDetector(kernel,"SiVertexBarrel");
  setupDetector(kernel,"SiVertexEndcap");
  // .... more subdetectors here .....
  setupDetector(kernel,"LumiCal");
  setupDetector(kernel,"BeamCal");

  kernel.configure();
  kernel.initialize();
  kernel.run();
  std::cout << "Successfully executed application .... " << std::endl;
  kernel.terminate();
}
\end{code}

\noindent
\begin{tabular} {l||p{0cm}}
\docline{Line}{}
\docline{1}{The header file \tts{Geant4Config.h} contains a set of wrapper
    classes to easy the creation of objects using the plugin mechanism and setting
    properties to \tts{Geant4Action} objects. These helpers and the corresponding 
    functionality are not included in the wrapped classes themselves to not 
    clutter the code with stuff only used for the setup.
    All contained objects are in the namespace \tts{dd4hep::sim::Setup}}.
\docline{6-10}{Save yourself specifying all the namespaces objects are in....}
\docline{13-19}{CINT processing pragmas. 
    Classes defined here will be available at the ROOT prompt
    after this AClick is loaded.}
\docline{22-29}{Sampler to fill the sensitive detector sequences for each 
    subdetector with two entries: a handler and a monitor action.
    Please note, that this here is example code and in real life specialized actions
    will have to be provided for each subdetector.}
\docline{31}{Let's go for it. here the entry point starts....}
\docline{32}{Create the \tts{Geant4Kernel} object.}
\docline{33}{Load the geometry into \DDhep.}
\docline{34}{Redefine the setup of the sensitive detectors.}
\docline{36-40}{Create the generator action of type \tts{Geant4ParticleGun} with name
    \tts{Gun}, set non-default properties and activate the configured object
    by attaching it to the \tts{Geant4Kernel}.}
\docline{42-46}{Create a user defined begin-of-run action callback, set the properties
    and attach it to the begin of run calls. To collect statistics extra member functions
    are registered to be called at the beginning and the end of each event.}
\docline{48-51}{Create a user defined event action routine, set its properties
    and attach it to the event action sequence.}
\docline{53-54}{Create a second event action and register it to the event action sequence.
    This action will be called after the previously created action.}
\docline{56-57}{For this event action we want to receive callbacks at start- 
    and end-of-run to produce additional summary output.}
\docline{59-63}{Call the sampler routine to attach test actions to the subdetectors defined.}
\docline{65-66}{Configure, initialize and run the Geant4 application.
    Most of the Geant4 actions will only be created here and the action sequences
    created before will be attached now.}
\docline{69}{Terminate the Geant4 application and exit.}
\end{tabular}

\newpage
\noindent
CINT currently cannot handle pointers to member functions~\footnote{This may change
in the future once ROOT uses \tts{clang} and \tts{cling} as the interpreting engine.}. 
Hence the above AClick only works in compiled mode. To invoke the compilation the following 
action is necessary from the ROOT prompt:


\begin{code}
$> root.exe
  *******************************************
  *                                         *
  *        W E L C O M E  to  R O O T       *
  *                                         *
  *   Version   5.34/10    29 August 2013   *
  *                                         *
  *  You are welcome to visit our Web site  *
  *          http://root.cern.ch            *
  *                                         *
  *******************************************

ROOT 5.34/10 (heads/v5-34-00-patches@v5-34-10-5-g0e8bac8, Sep 04 2013, 11:52:19 on linux)

CINT/ROOT C/C++ Interpreter version 5.18.00, July 2, 2010
Type ? for help. Commands must be C++ statements.
Enclose multiple statements between { }.
root [0] .X initAClick.C
.... Setting up the CINT include pathes and the link statements.

root [1] .L ../DD4hep.trunk/DDG4/examples/exampleAClick.C+
Info in <TUnixSystem::ACLiC>: creating shared library ....exampleAClick_C.so
.... some Cint warnings concerning member function pointers .....

root [2] exampleAClick()
.... and it starts ...
\end{code}

\noindent
The above scripts are present in the DDG4/example directory located in svn.
The initialization script \tts{initAClick.C} may require customization
to cope with the installation paths.

%=============================================================================
\subsection{Setting up DDG4 using Python}
\label{sec:ddg4-implementation-setup-python}
%=============================================================================

\noindent
Given the reflection interface of ROOT, the setup of the simulation interface
using dd4hep is of course also possible using the python interpreted language.
In the following code example the setup of Geant4 using the \tts{ClicSid} 
example is shown using python~\footnote{For comparison, the same example was 
used to illustrate the setup using XML files.}.

\begin{code}
import DDG4
from SystemOfUnits import *

"""

   dd4hep example setup using the python configuration

   @author  M.Frank
   @version 1.0

"""
def run():
  kernel = DDG4.Kernel()
  kernel.loadGeometry("file:../DD4hep.trunk/DDExamples/CLICSiD/compact/compact.xml")
  kernel.loadXML("DDG4_field.xml")

  description = kernel.detectorDescription()
  print '+++   List of sensitive detectors:'
  for i in description.detectors(): 
    o = DDG4.DetElement(i.second)
    sd = description.sensitiveDetector(o.name())
    if sd.isValid():
      print '+++  %-32s type:%s'%(o.name(), sd.type(), )

  # Configure Run actions
  run1 = DDG4.RunAction(kernel,'Geant4TestRunAction/RunInit')
  run1.Property_int    = 12345
  run1.Property_double = -5e15*keV
  run1.Property_string = 'Startrun: Hello_2'
  print run1.Property_string, run1.Property_double, run1.Property_int
  run1.enableUI()
  kernel.registerGlobalAction(run1)
  kernel.runAction().add(run1)

  # Configure Event actions
  evt2 = DDG4.EventAction(kernel,'Geant4TestEventAction/UserEvent_2')
  evt2.Property_int    = 123454321
  evt2.Property_double = 5e15*GeV
  evt2.Property_string = 'Hello_2 from the python setup'
  evt2.enableUI()
  kernel.registerGlobalAction(evt2)

  evt1 = DDG4.EventAction(kernel,'Geant4TestEventAction/UserEvent_1')
  evt1.Property_int=01234
  evt1.Property_double=1e11
  evt1.Property_string='Hello_1'
  evt1.enableUI()

  kernel.eventAction().add(evt1)
  kernel.eventAction().add(evt2)

  # Configure I/O
  evt_root = DDG4.EventAction(kernel,'Geant4Output2ROOT/RootOutput')
  evt_root.Control = True
  evt_root.Output = "simple.root"
  evt_root.enableUI()

  evt_lcio = DDG4.EventAction(kernel,'Geant4Output2LCIO/LcioOutput')
  evt_lcio.Output = "simple_lcio"
  evt_lcio.enableUI()

  kernel.eventAction().add(evt_root)
  kernel.eventAction().add(evt_lcio)

  # Setup particle gun
  gun = DDG4.GeneratorAction(kernel,"Geant4ParticleGun/Gun")
  gun.energy   = 0.5*GeV
  gun.particle = 'e-'
  gun.multiplicity = 1
  gun.enableUI()
  kernel.generatorAction().add(gun)

  # Setup global filters for use in sensitive detectors
  f1 = DDG4.Filter(kernel,'GeantinoRejectFilter/GeantinoRejector')
  f2 = DDG4.Filter(kernel,'ParticleRejectFilter/OpticalPhotonRejector')
  f2.particle = 'opticalphoton'
  f3 = DDG4.Filter(kernel,'ParticleSelectFilter/OpticalPhotonSelector') 
  f3.particle = 'opticalphoton'
  f4 = DDG4.Filter(kernel,'EnergyDepositMinimumCut')
  f4.Cut = 10*MeV
  f4.enableUI()
  kernel.registerGlobalFilter(f1)
  kernel.registerGlobalFilter(f2)
  kernel.registerGlobalFilter(f3)
  kernel.registerGlobalFilter(f4)

  # First the tracking detectors
  seq = DDG4.SensitiveSequence(kernel,'Geant4SensDetActionSequence/SiVertexBarrel')
  act = DDG4.SensitiveAction(kernel,'Geant4SimpleTrackerAction/SiVertexBarrelHandler','SiVertexBarrel')
  seq.add(act)
  seq.add(f1)
  seq.add(f4)
  act.add(f1)

  seq = DDG4.SensitiveSequence(kernel,'Geant4SensDetActionSequence/SiVertexEndcap')
  act = DDG4.SensitiveAction(kernel,'Geant4SimpleTrackerAction/SiVertexEndcapHandler','SiVertexEndcap')
  seq.add(act)
  seq.add(f1)
  seq.add(f4)

  seq = DDG4.SensitiveSequence(kernel,'Geant4SensDetActionSequence/SiTrackerBarrel')
  act = DDG4.SensitiveAction(kernel,'Geant4SimpleTrackerAction/SiTrackerBarrelHandler','SiTrackerBarrel')
  seq.add(act)
  seq.add(f1)
  seq.add(f4)

  seq = DDG4.SensitiveSequence(kernel,'Geant4SensDetActionSequence/SiTrackerEndcap')
  act = DDG4.SensitiveAction(kernel,'Geant4SimpleTrackerAction/SiTrackerEndcapHandler','SiTrackerEndcap')
  seq.add(act)

  seq = DDG4.SensitiveSequence(kernel,'Geant4SensDetActionSequence/SiTrackerForward')
  act = DDG4.SensitiveAction(kernel,'Geant4SimpleTrackerAction/SiTrackerForwardHandler','SiTrackerForward')
  seq.add(act)

  # Now the calorimeters
  seq = DDG4.SensitiveSequence(kernel,'Geant4SensDetActionSequence/EcalBarrel')
  act = DDG4.SensitiveAction(kernel,'Geant4SimpleCalorimeterAction/EcalBarrelHandler','EcalBarrel')
  seq.add(act)

  seq = DDG4.SensitiveSequence(kernel,'Geant4SensDetActionSequence/EcalEndcap')
  act = DDG4.SensitiveAction(kernel,'Geant4SimpleCalorimeterAction/EcalEndCapHandler','EcalEndcap')
  seq.add(act)

  seq = DDG4.SensitiveSequence(kernel,'Geant4SensDetActionSequence/HcalBarrel')
  act = DDG4.SensitiveAction(kernel,'Geant4SimpleCalorimeterAction/HcalBarrelHandler','HcalBarrel')
  act.adoptFilter(kernel.globalFilter('OpticalPhotonRejector'))
  seq.add(act)

  act = DDG4.SensitiveAction(kernel,'Geant4SimpleCalorimeterAction/HcalOpticalBarrelHandler','HcalBarrel')
  act.adoptFilter(kernel.globalFilter('OpticalPhotonSelector'))
  seq.add(act)

  seq = DDG4.SensitiveSequence(kernel,'Geant4SensDetActionSequence/HcalEndcap')
  act = DDG4.SensitiveAction(kernel,'Geant4SimpleCalorimeterAction/HcalEndcapHandler','HcalEndcap')
  seq.add(act)

  seq = DDG4.SensitiveSequence(kernel,'Geant4SensDetActionSequence/HcalPlug')
  act = DDG4.SensitiveAction(kernel,'Geant4SimpleCalorimeterAction/HcalPlugHandler','HcalPlug')
  seq.add(act)

  seq = DDG4.SensitiveSequence(kernel,'Geant4SensDetActionSequence/MuonBarrel')
  act = DDG4.SensitiveAction(kernel,'Geant4SimpleCalorimeterAction/MuonBarrelHandler','MuonBarrel')
  seq.add(act)

  seq = DDG4.SensitiveSequence(kernel,'Geant4SensDetActionSequence/MuonEndcap')
  act = DDG4.SensitiveAction(kernel,'Geant4SimpleCalorimeterAction/MuonEndcapHandler','MuonEndcap')
  seq.add(act)

  seq = DDG4.SensitiveSequence(kernel,'Geant4SensDetActionSequence/LumiCal')
  act = DDG4.SensitiveAction(kernel,'Geant4SimpleCalorimeterAction/LumiCalHandler','LumiCal')
  seq.add(act)

  seq = DDG4.SensitiveSequence(kernel,'Geant4SensDetActionSequence/BeamCal')
  act = DDG4.SensitiveAction(kernel,'Geant4SimpleCalorimeterAction/BeamCalHandler','BeamCal')
  seq.add(act)

  # Now build the physics list:
  phys = kernel.physicsList()
  phys.extends = 'FTFP_BERT'
  #phys.transportation = True
  phys.decays  = True
  phys.enableUI()

  ph = DDG4.PhysicsList(kernel,'Geant4PhysicsList/Myphysics')
  ph.addParticleConstructor('G4BosonConstructor')
  ph.addParticleConstructor('G4LeptonConstructor')
  ph.addParticleProcess('e[+-]','G4eMultipleScattering',-1,1,1)
  ph.addPhysicsConstructor('G4OpticalPhysics')
  ph.enableUI()
  phys.add(ph)

  phys.dump()

  kernel.configure()
  kernel.initialize()
  kernel.run()
  kernel.terminate()

if __name__ == "__main__":
  run()

\end{code}

\newpage
%=============================================================================
\subsection{A Simple Example}
\label{sec:ddg4-implementation-simple-example}
%=============================================================================
\noindent
Bla-bal.

\newpage

