%=============================================================================
\documentclass[10pt,a4paper]{article}
%
%
%
\usepackage{graphicx}
\usepackage{hyperref}
\usepackage{verbatim}
\usepackage{fix-cm}
\usepackage{lineno}
\usepackage{fancyhdr}
%
%
\oddsidemargin  0.1 in
\evensidemargin 0.1 in
%
%
\newlength{\backindent}\setlength{\backindent}{2cm}
\textwidth 5.375 in % Width of text line.
\advance\textheight by1.4cm
\advance\voffset by-1.4cm
\advance\textwidth by\backindent
%
%
% === Fancy headers setup  ===============================
%
\setlength{\headheight}{15.2pt}
\pagestyle{fancyplain} {
\fancyhead[L]{\includegraphics[height=10mm]{DD4hep-AIDA-logo.png}\vspace{-0.3cm}}
\fancyhead[C]{}
\fancyhead[R]{\sffamily{\underline{\hspace{6cm}Advanced European Infrastructures for Detectors at Accelerators}}}
\fancyfoot[L]{}
\fancyfoot[C]{\sffamily{User Manual}}
\fancyfoot[R]{\sffamily{\thepage}}
}
%
%
\newcommand{\tw}[1]{${\tt{#1}}$}
\newcommand{\tts}[1]{{\tt\small{#1}}}
\newcommand{\bold}[1]{{\bf{#1}}}
%
%
\newcommand{\DDE}{{$\tt{DDEve}$\space}}
\newcommand{\DDhep}{{$\tt{DD4hep}$\space}}
\newcommand{\DDH}{{$\tt{DD4hep}$\space}}
\newcommand{\DDG}{{\tt{DDG4}\space}}
\newcommand{\DDA}{{\tt{DDAlign}\space}}
\newcommand{\DDR}{{\tt{DDRec}\space}}
%%
%
\newcommand{\docline}[2]{\vspace{0.1cm}{\bf{#1}} & \parbox{14.5cm}{#2}\\}
%
% === Specialization of the lineno package
%
\renewcommand{\linenumberfont} {\normalfont\small\sffamily}
\renewcommand{\makeLineNumber} {\makeLineNumberLeft}
\renewcommand{\linenumbersep} {2pt}
%
% === Set font to code section with line numbers
%
\newenvironment{code}{\par\vspace{0.01cm}\small\linenumbers\verbatim\setcounter{linenumber}{1}}{\endverbatim\nolinenumbers\vspace{-0.02cm}}%
%
% === Set font to code section with line numbers
%
\newenvironment{unnumberedcode}{\par\vspace{-0.1cm}\small\verbatim\setcounter{linenumber}{1}}%
{\endverbatim\vspace{-0.2cm}}
%
% === Command to insert http links to the DD4hep geomtery package
%
\newcommand{\detdesc}[2]
{
    \href{http://www.cern.ch/frankm/DD4hep/doc/#1}{#2}
}
%
% === Command to insert http links to the ROOT geomtery package
%
\newcommand{\tgeo}[2]
{
    \href{http://root.cern.ch/root/html/#1.html}{#2}
}
%
% ===  Compactify the item list  =========================
%
\newcommand{\itemcompact}{\setlength{\itemsep}{1pt}\setlength{\parskip}{0pt}\setlength{\parsep}{0pt}}
%
%
% ===  Title page command  ===============================
%
\newcommand{\mytitle}[3]{
\begin{titlepage}
%
\pagestyle{empty}
%
\includegraphics[height=25mm] {DD4hep-AIDA-logo.png}

\vspace{0.02cm}

{\sffamily{\underline{\hspace{6cm}Advanced European Infrastructures for Detectors at Accelerators}}}

\vspace{2cm}

\begin{center}
{\fontsize{72}{32}\selectfont{\bfseries{#1}}}

\vspace{3cm}
{\Huge\bf{#2}}

\vspace{5cm}
{#3}
%M.Frank%\textsuperscript{1},
%F.Gaede\textsuperscript{2},
%C.Grefe\textsuperscript{1},
%P.Mato\textsuperscript{1}
%{\textsuperscript{1} 
%{CERN, 1211 Geneva 23, Switzerland}
%{\textsuperscript{2} Desy, 22607 Hamburg, Germany}
\end{center}
\end{titlepage}
}

%
%
\usepackage{graphicx}
\usepackage{hyperref}
\usepackage{verbatim}
\usepackage{fix-cm}
\usepackage{setup/lineno}
\usepackage{fancyhdr}
%\usepackage{amsmath}
%
\oddsidemargin  0.1 in
\evensidemargin 0.1 in
%
%
\newlength{\backindent}\setlength{\backindent}{2cm}
\textwidth 5.375 in % Width of text line.
\advance\textheight by1.4cm
\advance\voffset by-1.4cm
\advance\textwidth by\backindent
%
%
% === Fancy headers setup  ===============================
%
\setlength{\headheight}{15.2pt}
\pagestyle{fancyplain} {
\fancyhead[L]{\includegraphics[height=10mm]{setup/AIDA2020-logo.png}\vspace{-0.3cm}}
\fancyhead[C]{}
\fancyhead[R]{\sffamily{\underline{\hspace{6cm}Advanced European Infrastructures for Detectors at Accelerators}}}
\fancyfoot[L]{}
\fancyfoot[C]{\sffamily{User Manual}}
\fancyfoot[R]{\sffamily{\thepage}}
}
%
%
\newcommand{\tw}[1]{${\tt{#1}}$}
\newcommand{\tts}[1]{{\tt\small{#1}}}
\newcommand{\bold}[1]{{\bf{#1}}}
%
%
\newcommand{\docline}[2]{\vspace{0.1cm}{\bf{#1}} & \parbox{14.5cm}{#2}\\}
%
% === Specialization of the lineno package
%
\renewcommand{\linenumberfont} {\normalfont\small\sffamily}
\renewcommand{\makeLineNumber} {\makeLineNumberLeft}
\renewcommand{\linenumbersep} {2pt}
%
% === Set font to code section with line numbers
%
\newenvironment{code}{\par\vspace{0.01cm}\small\linenumbers\verbatim\setcounter{linenumber}{1}}{\endverbatim\nolinenumbers\vspace{-0.02cm}}%
%
% === Set font to code section with line numbers
%
\newenvironment{unnumberedcode}{\par\vspace{-0.1cm}\small\verbatim\setcounter{linenumber}{1}}%
{\endverbatim\vspace{-0.2cm}}
%
%
% ===  Compactify the item list  =========================
%
\newcommand{\itemcompact}{\setlength{\itemsep}{1pt}\setlength{\parskip}{0pt}\setlength{\parsep}{0pt}}
%
%
% ===  Title page command  ===============================
%
%
\newcommand{\basictitle}[2]{
%
\pagestyle{empty}
%
\includegraphics[height=25mm] {setup/AIDA2020-logo.png}

\vspace{0.02cm}

{\sffamily{\underline{\hspace{6cm}Advanced European Infrastructures for Detectors at Accelerators}}}

\vspace{2cm}

\begin{center}
{\fontsize{72}{32}\selectfont{\bfseries{#1}}}

\vspace{3cm}
{\Huge\bf{#2}}
\vspace{3cm}
\begin{figure}[b]
  \begin{center}
    \includegraphics[height=15mm] {setup/Horizon2020-grant-logo.png}
  \end{center}
\end{figure}
\end{center}
}
\newcommand{\AIDAtitle}[3]{
\begin{titlepage}
\basictitle{#1}{#2}
\begin{center}
{#3}
\end{center}
\end{titlepage}
}

%
\pagestyle{fancyplain}{\fancyfoot[C]{\sffamily{DDG4 User Manual}}}
%
\begin{document}   
%
\mytitle{DDG4}   % Abbreviated title
{  % Detailed title
A Simulation Toolkit for \\
\vspace{0.5cm}
High Energy Physics Experiments\\
\vspace{0.5cm}
using Geant4 and the \\
\vspace{0.5cm}
DD4hep Geometry Description\\
}
{  % Author list
M. Frank \\
{CERN, 1211 Geneva 23, Switzerland}
}
%
%
%==  Abstract  ===============================================================
\pagestyle{plain}
\pagenumbering{Roman}
\setcounter{page}{1}
\begin{abstract}
%=============================================================================

\noindent
\normalsize
Simulating the detector response is an essential tool in high energy physics
to analyze the sensitivity of an experiment to the underlying physics.
Such simulation tools require a detailed though convenient detector description as 
it is provided by the \DDhep toolkit.
We will present the generic simulation toolkit \DDG using the \DDhep detector 
description toolkit. 
The toolkit implements a modular and flexible approach to simulation activities
using Geant4. User defined simulation applications using \DDG 
can easily be configured, extended using specialized action routines.
The design is strongly driven by easy of use;
developers of detector descriptions and applications using
them should provide minimal information and minimal specific
code to achieve the desired result.

\end{abstract}

\vspace{10cm}

\begin{center}
{\large{\bf{
\begin{tabular} {| l | l | l |}
\hline
\multicolumn{3}{| c |}{} \\[0.2cm]
\multicolumn{3}{| c |}{Document History} \\[0.2cm]
\multicolumn{3}{| c |}{} \\[0.2cm]
\hline
                 &      &        \\
Document         &      &        \\
version          & Date & Author \\[0.2cm] \hline
                 &      &        \\
1.0              & 19/11/2013 & Markus Frank CERN/LHCb  \\
                 &      &        \\        \hline 
\end{tabular}
}}}
\end{center}

\clearpage
%
%
%==  TOC  ====================================================================
\tableofcontents
\clearpage
%
%
%=============================================================================
% Manual
%=============================================================================
\pagenumbering{arabic}
\setcounter{page}{1}
\graphicspath{{./figs/}}
\input{sections/Introduction.tex}

%=============================================================================
\section{DDG4 Implementation}
\label{sec:ddg4-user-manual-implementation}
%=============================================================================

\noindent
A basic design criteria of the a \DDG simulation application was to 
process any user defined hook provided by Geant4 as a series of algorithmic
procedures, which could be implemented either using inheritance or by 
a callback mechanism registering functions fulfilling a given signature.
Such sequences are provided for all actions mentioned in the list in 
Section~\ref{sec:ddg4-user-manual-geant4-interface} as well as for 
the callbacks to sensitive detectors.

\noindent
The callback mechanism was introduced to allow for weak coupling between 
the various actions. For example could an action performing monitoring
using histograms at the event level initialize or reset its histograms
at the start/end of each run. To do so, clearly a callback at the 
start/end of a run would be necessary.

\noindent
In the following sections a flexible and extensible interface to hooks
of Geant4 is discussed starting with the description of the basic
components \tts{Geant4Kernel} and \tts{Geant4Action} followed by the 
implementation of the relevant specializations.
The specializations exposed are sequences of such actions,
which also call registered objects.
In later section the configuration and the combination of these components 
forming a functional simulation application is presented.

%=============================================================================
\subsection{The Application Core Object: Geant4Kernel}
\label{sec:ddg4-user-manual-implementation-geant4kernel}
%=============================================================================

\noindent
The kernel object is the central context of a \DDG simulation application and
gives all clients access to the user hooks (see Figure~\ref{fig:ddg4-geant4-kernel}).
All Geant4 callback structures are exposed so that clients can easily 
objects implementing the required interface or register callbacks with the 
correct signature. Each of these action sequences is connected to an instance
of a Geant4 provided callback structure as it is shown in
Figure~\ref{fig:ddg4-g4runmanager-anatomy}.
\begin{figure}[h]
  \begin{center}
    \includegraphics[height=65mm] {DDG4-Geant4Kernel}
    \caption{The main application object gives access to all sequencing actions
    in a \DDG4 application. Sequence actions are only container of user actions
    calling one user action after the other. Optionally single callbacks may 
    be registered to a user action.}
    \label{fig:ddg4-geant4-kernel}
  \end{center}
\end{figure}

%=============================================================================
\subsection{Action Sequences}
\label{sec:ddg4-user-manual-implementation-geant4action-sequences}
%=============================================================================

\noindent
As shown in 

%=============================================================================
\subsection{The Base Class of DDG4 Actions: Geant4Action}
\label{sec:ddg4-user-manual-implementation-geant4action-base}
%=============================================================================

\noindent
The class \tts{Geant4Action} is a common component interface providing 
the basic interface to the framework to
\begin{itemize}\itemcompact
\item configure the component using a property mechanism
\item provide an appropriate interface to Geant4 interactivity. The interactivity 
    included a generic way to change and access properties from the Geant4 UI 
    prompt as well as executing registered commands.
\item As shown in Figure~\ref{fig:ddg4-implementation-geant4-action}, the 
    base class also provides to its sub-class a reference to the \tts{Geant4Kernel}
    objects through the \tts{Geant4Context}.
\end{itemize}
The \tts{Geant4Action} is a named entity and can be uniquely identified within
a sequence attached to one Geant4 user callback.
%=============================================================================
\begin{figure}[h]
  \begin{center}
    \includegraphics[height=30mm] {DDG4-Geant4Action}
    \caption{The design of the common base class \tts{Geant4Action}.}
    \label{fig:ddg4-implementation-geant4-action}
  \end{center}
\end{figure}

\noindent
\DDG knows two types of actions: global actions and anonymous actions.
Global actions are accessible externally from the \tts{Geant4Kernel} instance.
Global actions are also re-usable and hence may be contribute to several 
action sequences (see the following chapters for details). Global actions 
are uniquely identified by their name.
Anonymous actions are known only within one sequence and normally
are not shared between sequences.

%=============================================================================
\subsubsection{The Properties of \bold{Geant4Action} Instances}
\label{sec:ddg4-implementation-geant4-action-properties}
%=============================================================================

\noindent
Nearly any subclass of a \tts{Geant4Action} needs a flexible configuration 
in order to be reused, modified etc. The implementation of the mechanism
uses a very flexible value conversion mechanism using \tts{boost::spirit},
which support also conversions between unrelated types provided a dictionary 
is present.

\noindent
Properties are supposed to be member variables of a given action object.
To publish a property it needs to be declared in the constructor as shown here:
\begin{unnumberedcode}
  declareProperty("OutputLevel", m_outputLevel = INFO);
  declareProperty("Control",     m_needsControl = false);
\end{unnumberedcode}
The internal setup of the \tts{Geant4Action} objects then ensure that 
all declared properties will be set after the object construction to the 
values set in the setup file.

\noindent
\bold{Note:} Because the values can only be set \bold{after} the object 
was constructed, the actual values may not be used in the constructor
of any base or sub-class.

%=============================================================================
\subsection{Geant4 Action Sequences}
\label{sec:ddg4-user-manual-implementation-geant4action-sequences}
%=============================================================================

\noindent
All Geant4 user hooks are realized as action sequences. As shown in 
Figure~\ref{fig:ddg4-geant4-kernel} these sequences are accessible to the user,
who may attach specialized actions to the different action sequences. This 
allows a flexible handing of specialized user actions e.g. to dynamically
add monitoring actions filling histograms or to implement alternative hit 
creation mechanism in a sensitive detector for detailed detector studies.
Figure~\ref{fig:ddg4-implementation-sequence-calls} shows the schematic
call structure of an example {\tt{Geant4TrackingActionSequence}}:\\
\begin{figure}[h]
  \begin{center}
    \includegraphics[width=150mm] {DDG4-TrackingActionCalls}
    \caption{The design of the tracking action sequence. Specialized 
               tracking action objects inherit from the \tts{Geant4TrackingAction}
               object and must be attached to the sequence.}
    \label{fig:ddg4-implementation-sequence-calls}
  \end{center}
\end{figure}

\noindent
Geant4 calls the function from the virtual interface (\tts{G4UserTrackingAction}), 
which is realised by the \tts{Geant4UserTrackingAction} with the single purpose to
propagate the call to the action sequence, which then calls all registered clients
of type \tts{Geant4TrackingAction}.

\noindent
The main action sequences have a fixed name. These are
\begin{itemize}

\item The \bold{RunAction} attached to the \tts{G4UserRunAction}, implemented 
    by the \tts{Geant4RunActionSequence} class and is called at the start and the end of 
    every run (beamOn). Members of the \tts{Geant4RunActionSequence} are of type
    \tts{Geant4RunAction} and receive the callbacks by overloading the two routines:
\begin{unnumberedcode}
/// begin-of-run callback
virtual void begin(const G4Run* run);
/// End-of-run callback
virtual void end(const G4Run* run);
\end{unnumberedcode}
    or register a callback with the signature {\tts{void (T::*)(const G4Run*)}}
    either to receive begin-of-run or end-or-calls using the methods:
\begin{unnumberedcode}
/// Register begin-of-run callback. Types Q and T must be polymorph!
template <typename Q, typename T> void callAtBegin(Q* p, void (T::*f)(const G4Run*));
/// Register end-of-run callback. Types Q and T must be polymorph!
template <typename Q, typename T> void callAtEnd(Q* p, void (T::*f)(const G4Run*));
\end{unnumberedcode}
    of the \tts{Geant4RunActionSequence} from the \tts{Geant4Context} object.


\item The \bold{EventAction} attached to the \tts{G4UserEventAction}, implemented 
    by the \tts{EventActionSequence} class and is called at the start and the end of 
    every event. Members of the \tts{Geant4EventActionSequence} are of type
    \tts{Geant4EventAction} and receive the callbacks by overloading the two routines:
\begin{unnumberedcode}
/// Begin-of-event callback
virtual void begin(const G4Event* event);
/// End-of-event callback
virtual void end(const G4Event* event);
\end{unnumberedcode}
    or register a callback with the signature {\tts{void (T::*)(const G4Event*)}}
    either to receive begin-of-run or end-or-calls using the methods:
\begin{unnumberedcode}
/// Register begin-of-event callback
template <typename Q, typename T> void callAtBegin(Q* p, void (T::*f)(const G4Event*));
/// Register end-of-event callback
template <typename Q, typename T> void callAtEnd(Q* p, void (T::*f)(const G4Event*));
\end{unnumberedcode}
    of the \tts{Geant4EventActionSequence} from the \tts{Geant4Context} object.


\item The \bold{GeneratorAction} attached to the \tts{G4VUserPrimaryGeneratorAction}, implemented 
    by the \tts{Geant4GeneratorActionSequence} class and is called at the start of 
    every event and provided all initial tracks from the Monte-Carlo generator.
    Members of the \tts{Geant4GeneratorActionSequence} are of type
    \tts{Geant4EventAction} and receive the callbacks by overloading the member function:
\begin{unnumberedcode}
/// Callback to generate primary particles
virtual void operator()(G4Event* event);
\end{unnumberedcode}
    or register a callback with the signature {\tts{void (T::*)(G4Event*)}}
    to receive calls using the method:
\begin{unnumberedcode}
/// Register primary particle generation callback.
template <typename Q, typename T> void call(Q* p, void (T::*f)(G4Event*));
\end{unnumberedcode}
    of the \tts{Geant4GeneratorActionSequence} from the \tts{Geant4Context} object.

\end{itemize}
\begin{figure}[t]
  \begin{center}
    \includegraphics[width=160mm] {DDG4-TrackingAction}
    \caption{The design of the tracking action sequence. Specialized 
               tracking action objects inherit from the \tts{Geant4TrackingAction}
               object and must be attached to the sequence.}
    \label{fig:ddg4-implementation-tracking-action}
  \end{center}
\end{figure}

\begin{itemize}
\item The \bold{TrackingAction} attached to the \tts{G4UserTrackingAction}, 
    implemented by the \tts{Geant4-} \tts{Tracking\-Action\-Sequence} class 
    and is called at the start and the end of tracking one single particle 
    trace through the material of the detector.
    Members of the \tts{Geant4\-Tracking\-ActionSequence} are of type
    \tts{Geant4TrackingAction} and receive the callbacks by overloading the member function:
\begin{unnumberedcode}
/// Pre-tracking action callback
virtual void begin(const G4Track* trk);
/// Post-tracking action callback
virtual void end(const G4Track* trk);
\end{unnumberedcode}
    or register a callback with the signature {\tts{void (T::*)(const G4Step*, G4SteppingManager*)}}
    to receive calls using the method:
\begin{unnumberedcode}
/// Register Pre-track action callback
template <typename Q, typename T> void callAtBegin(Q* p, void (T::*f)(const G4Track*));
/// Register Post-track action callback
template <typename Q, typename T> void callAtEnd(Q* p, void (T::*f)(const G4Track*));
\end{unnumberedcode}
Figure~\ref{fig:ddg4-implementation-tracking-action} show as an example 
the design (class-diagram) of the \tts{Geant4TrackingAction}.


\item The \bold{SteppingAction} attached to the \tts{G4UserSteppingAction}, implemented 
    by the \tts{Geant4-} \tts{SteppingActionSequence} class and is called for each
    step when tracking a particle.
    Members of the \tts{Geant4SteppingActionSequence} are of type
    \tts{Geant4SteppingAction} and receive the callbacks by overloading the member function:
\begin{unnumberedcode}
/// User stepping callback
virtual void operator()(const G4Step* step, G4SteppingManager* mgr);
\end{unnumberedcode}
    or register a callback with the signature {\tts{void (T::*)(const G4Step*, G4SteppingManager*)}}
    to receive calls using the method:
\begin{unnumberedcode}
/// Register stepping action callback.
template <typename Q, typename T> void call(Q* p, void (T::*f)(const G4Step*, 
                                                               G4SteppingManager*));
\end{unnumberedcode}


\item The \bold{StackingAction} attached to the 
    {\tts{G4UserStackingAction}}, implemented by the \tts{Geant4-}\\
    \tts{StackingActionSequence} class.
    Members of the \tts{Geant4StackingActionSequence} are of type\\
    \detdesc{html/class_d_d4hep_1_1_simulation_1_1_geant4_stacking_action.html}
    {\tts{Geant4StackingAction}} and receive the callbacks by overloading the member functions:
\begin{unnumberedcode}
/// New-stage callback
virtual void newStage();
/// Preparation callback
virtual void prepare();
\end{unnumberedcode}
    or register a callback with the signature {\tts{void (T::*)()}}
    to receive calls using the method:
\begin{unnumberedcode}
/// Register begin-of-event callback. Types Q and T must be polymorph!
template <typename T> void callAtNewStage(T* p, void (T::*f)());
/// Register end-of-event callback. Types Q and T must be polymorph!
template <typename T> void callAtPrepare(T* p, void (T::*f)());
\end{unnumberedcode}
\end{itemize}

\noindent
All sequence types support the method \tts{void adopt(T* member\_reference)}
to add the members. Once adopted, the sequence takes ownership and manages
the member. The design of all sequences is very similar. 

%=============================================================================
\subsection{Sensitive Detectors}
\label{sec:ddg4-user-manual-geant4sensitivedetectors}
%=============================================================================

\noindent
Sensitive detectors are associated by the detector designers to all active 
materials, which would produce a signal which can be read out. In Geant4 this concept
is realized by using a base class \tts{G4VSensitiveDetector}.
The mandate of a sensitive detector is the construction of hit objects 
using information from steps along a particle track. 
The \tts{G4VSensitiveDetector} receives 
a callback at the begin and the end of the event processing and at each step
inside the active material whenever an energy deposition occurred.

\begin{figure}[t]
  \begin{center}
    \includegraphics[height=110mm] {DDG4-Sensitive-detector}
    \caption{The sensitive detector design. The actual energy deposits are 
        collected in user defined subclasses of the \tts{Geant4Sensitive}.
        Here, as an example possible actions called \tts{TrackerHitCollector},
        \tts{TrackerDetailedHitCollector} and \tts{TrackerHitMonitor} are shown.}
    \label{fig:ddg4-implementation-sensitive-detector}
  \end{center}
\end{figure}

\noindent
The sensitive actions do not necessarily deal only the collection of energy 
deposits, but could also be used to simply monitor the performance of the
active element e.g. by producing histograms of the absolute value or the 
spacial distribution of the depositions.

\noindent
Within \DDG the concept of sensitive  detectors is implemented as a
configurable  action sequence of type 
\detdesc{html/class_d_d4hep_1_1_simulation_1_1_geant4_sens_det_action_sequence.html}
{\tts{Geant4SensDetActionSequence}}
calling members of the type 
\detdesc{html/struct_d_d4hep_1_1_simulation_1_1_geant4_sensitive.html}
{\tts{Geant4Sensitive}} as shown in 
Figure~\ref{fig:ddg4-implementation-sensitive-detector}. The actual processing
part of such a sensitive action is only called if the and of a set of
required filters of type \tts{Geant4Filter} is positive (see also 
section~\ref{sec:ddg4-implementation-sensitive-detector-filters}). No filter 
is also positive. Possible filters are e.g. particle filters, which ignore the
sensitive detector action if the particle is a \tts{geantino} or if the
energy deposit is below a given threshold.

\noindent
Objects of type \tts{Geant4Sensitive} receive the callbacks by overloading the 
member function:
\begin{unnumberedcode}
  /// Method invoked at the beginning of each event.
  virtual void begin(G4HCofThisEvent* hce);
  /// Method invoked at the end of each event.
  virtual void end(G4HCofThisEvent* hce);
  /// Method for generating hit(s) using the information of G4Step object.
  virtual bool process(G4Step* step, G4TouchableHistory* history);
  /// Method invoked if the event was aborted.
  virtual void clear(G4HCofThisEvent* hce);
\end{unnumberedcode}
or register a callback with the signature {\tts{void (T::*)(G4HCofThisEvent*)}}
respectively {\tts{void (T::*)(G4Step*, G4TouchableHistory*)}} 
to receive callbacks using the methods:
\begin{unnumberedcode}
  /// Register begin-of-event callback
  template <typename T> void callAtBegin(T* p, void (T::*f)(G4HCofThisEvent*));
  /// Register end-of-event callback
  template <typename T> void callAtEnd(T* p, void (T::*f)(G4HCofThisEvent*));
  /// Register process-hit callback
  template <typename T> void callAtProcess(T* p, void (T::*f)(G4Step*, G4TouchableHistory*));
  /// Register clear callback
  template <typename T> void callAtClear(T* p, void (T::*f)(G4HCofThisEvent*));
\end{unnumberedcode}
Please refer to the Geant4 Applications manual from the Geant4 web page for 
further details about the concept of sensitive detectors.

%=============================================================================
\subsubsection{Helpers of Sensitive Detectors: The Geant4VolumeManager}
\label{sec:ddg4-user-manual-geant4volumemanager}%=============================================================================

\noindent
Sooner or later, when a hit is created in a sensitive placed volume, the
hit must be associated with this volume. For this purpose \DDhep provides 
the concept of the \tts{VolumeManager}, which identifies placed volumes uniquely 
by a 64-bit identifier, the $CellID$. This mechanism allows to quickly
retrieve a given volume given the hit data containing the $CellID$.
The $CellID$ is a very compressed representation for any element in the 
hierarchy of placed volumes to the sensitive volume in question.

\noindent 
During the simulation the reverse mechanism must be applied: Geant4 provides
the hierarchy of \tts{G4PhysicalVolumes} to the hit location and the local coordinates
of the hit within the sensitive volume. Hence to determine the volume identifier
is essential to store hits so that they can be later accessed and processed efficiently.
This mechanism is provided by the \tts{Geant4VolumeManager}. Clients typically do not
interact with this object, any access necessary is provided by the
\tts{Geant4Sensitive} action:
\begin{unnumberedcode}
  /// Method for generating hit(s) using the information of G4Step object.
  bool MySensitiveAction:process(G4Step* step,G4TouchableHistory* /*hist*/ ) {
    ...
    Hit* hit = new Hit();
    // *** Retrieve the cellID  ***
    hit->cellID = cellID(step);
    ...
  }
\end{unnumberedcode}
The call is realized using a member function provided by the 
\tts{Geant4Sensitive} action:
\begin{unnumberedcode}
  /// Returns the cellID of the sensitive volume corresponding to the step
  /** The CellID is the VolumeID + the local coordinates of the sensitive area.
   *  Calculated by combining the VolIDS of the complete geometry path (Geant4TouchableHistory)
   *  from the current sensitive volume to the world volume
   */
  long long int cellID(G4Step* step);
\end{unnumberedcode}

\noindent
\bold{Note:}\\
The \tts{Geant4VolumeManager} functionality is not for free! It requires that


\noindent
-- match Geant4 volume with TGeo volume

%=============================================================================
\subsubsection{DDG4 Intrinsic Sensitive Detectors}
%=============================================================================
\noindent
Currently there are two generic sensitive detectors implemented in DDG4:
\begin{itemize}\itemcompact
\item The \tts{Geant4TrackerAction}, which may be used to handle tracking devices.
  This sensitive detector produces one hit for every energy deposition of Geant4
  i.e. for every callback to 
\begin{unnumberedcode}
  /// Method for generating hit(s) using the information of G4Step object.
  virtual bool process(G4Step* step, G4TouchableHistory* history);
\end{unnumberedcode}
  See the implementation file 
  \detdesc{html/_geant4_s_d_actions_8cpp_source.html}{DDG4/plugins/Geant4SDAction.cpp}
  for details. The produced hits are of type 
  \detdesc{html/_geant4_data_8h_source.html}{Geant4Tracker::Hit}.

\item The \tts{Geant4CalorimeterAction}, which may be used to handle 
  generic calorimeter like devices.
  This sensitive detector produces at most one hit for every cell in the calorimeter.
  If several tracks contribute to the energy deposit of this cell, the contributions
  are added up.
  See the implementation file 
  \detdesc{html/_geant4_s_d_actions_8cpp_source.html}{DDG4/plugins/Geant4SDAction.cpp}
  for details. The produced hits are of type 
  \detdesc{html/_geant4_data_8h_source.html}{Geant4Calorimeter::Hit}.
  propagate the MC truth information with respect to each track kept in the 
  particle record.
\end{itemize}

\noindent
Both sensitive detectors use the \tts{Geant4VolumeManager} discussed in 
section~\ref{sec:ddg4-user-manual-geant4volumemanager} to identify the sensitive elements.

\noindent
\bold{PLEASE NOTE:}\\
The above palette of generic sensitive detectors only contains two very
often used implementations. We hope, that this palette over time grows from
external contributions of other generic sensitive detectors. We would be happy 
to extend this palette with other generic implementations. One example would
be the handling of the simulation response for optical detectors like RICH-Cerenkov
detectors.

%=============================================================================
\subsubsection{Sensitive Detector Filters}
\label{sec:ddg4-implementation-sensitive-detector-filters}
%=============================================================================

\noindent
The concept of filters allows to build more flexible sensitive detectors by
restricting the hit processing of a given instance of a sensitive action.

\begin{itemize}\itemcompact
\item Examples would be to demand a given particle type before a sensitive action is 
invoked: a sensitive action dealing with optical photons (RICH detectors, etc),
would e.g. not be interested in energy depositions of other particles.
A filter object restricting the particle type to optical photons would 
be appropriate.
\item Another example would be to implement a special action instance, which would
be only called if the filter requires a minimum energy deposit.
\end{itemize}
There are plenty of possible applications, hence we would like 
to introduce this feature here.

\noindent
Filters are called by Geant4 before the
hit processing in the sensitive detectors start. The global filters
may be shared between many sensitive detectors. Alternatively filters
may be directly attached to the sensitive detector in question.
Attributes are directly passed as properties to the filter action.

\noindent
Technically do \tts{Geant4Filter} objects inherit from the base class
\tts{Geant4Filter} (see Figure~\ref{fig:ddg4-implementation-sensitive-detector-filters}.
Any filter inherits from the common base class \tts{Geant4Filter}, then 
several specializations may be configured like filters to select/reject 
particles, to specify the minimal energy deposit to be processed etc.
A sensitive detector is called if the filter callback with the signature
returns a true result:
\begin{unnumberedcode}
  /// Filter action. Return true if hits should be processed
  virtual bool operator()(const G4Step* step) const;
\end{unnumberedcode}
\begin{figure}[h]
  \begin{center}
    \includegraphics[height=65mm] {DDG4-SensitiveFilterClasses}
    \caption{The sensitive detector filters design. The shown class
        diagram is actually implemented.}
    \label{fig:ddg4-implementation-sensitive-detector-filters}
  \end{center}
\end{figure}

\newpage

%=============================================================================
\subsection{The Geant4 Physics List}
\label{sec:ddg4-implementation-physics-list}
%=============================================================================
\noindent 
Geant4 provides the base class \tts{G4VUserPhysicsList}, which allows users
to implement customized physics according to the studies to be made.
Any user defined physics list must provide this interface. DDG4 provides such an interface
through the ROOT plugin mechanism using the class \tts{G4VModularPhysicsList}.
The flexibility of \DDG allows for several possibilities to setup the Geant4
physics list. Instead of explicitly coding the physics list, \DDG foresees the
usage of the plugin mechanism to instantiate the necessary calls to Geant4 in a
sequence of actions:
\begin{itemize}
\item The \bold{physics list} is realized as a sequence of actions of type 
    \detdesc{html/class_d_d4hep_1_1_simulation_1_1_geant4_physics_list_action_sequence.html}
    {\tts{Geant4PhysicsListActionSequence}}.
    Members of the \detdesc{html/class_d_d4hep_1_1_simulation_1_1_geant4_physics_list_action_sequence.html}
    {\tts{Geant4PhysicsListActionSequence}} are of type
    \detdesc{html/class_d_d4hep_1_1_simulation_1_1_geant4_physics_list.html}
    {\tts{Geant4PhysicsList}} and receive the callbacks by overloading 
    the member functions:
\begin{unnumberedcode}
  /// Callback to construct the physics constructors
  virtual void constructProcess(Geant4UserPhysics* interface);
  /// constructParticle callback
  virtual void constructParticles(Geant4UserPhysics* particle);
  /// constructPhysics callback
  virtual void constructPhysics(Geant4UserPhysics* physics);
\end{unnumberedcode}
    or register a callback with the signature {\tts{void (T::*)(Geant4UserPhysics*)}}
    to receive calls using the method:
\begin{unnumberedcode}
  /// Register process construction callback t
  template <typename Q, typename T> void constructProcess(Q* p, void (T::*f)(Geant4UserPhysics*));
  /// Register particle construction callback
  template <typename Q, typename T> void constructParticle(Q* p, void (T::*f)(Geant4UserPhysics*));
\end{unnumberedcode}
    The argument of type \detdesc{html/class_d_d4hep_1_1_simulation_1_1_geant4_user_physics.html}
    {\tts{Geant4UserPhysics}} provides a basic interface to the original
    \tts{G4VModular}- \tts{PhysicsList}, which allows to register physics constructors etc.

\item In most of the cases the above approach is an overkill and often even too flexible.
    Hence, alternatively, the physics list may consist of a single entry of type 
    \detdesc{html/class_d_d4hep_1_1_simulation_1_1_geant4_physics_list.html}
    {\tts{Geant4PhysicsList}}.
\end{itemize}

\noindent
The basic implementation of the \tts{Geant4PhysicsList} supports the usage of various
\begin{itemize}\itemcompact
\item \detdesc{html/_geant4_particles_8cpp_source.html}{particle constructors}, 
    such as single particle constructors like   
    \tts{G4Gamma} or \tts{G4Proton}, or whole particle groups like
    \tts{G4BosonConstructor} or \tts{G4IonConstrutor},
\item \detdesc{html/_geant4_processes_8cpp_source.html}{physics process constructors}, 
    such as e.g. \tts{G4GammaConversion},
    \tts{G4PhotoElectricEffect} or\\ \tts{G4ComptonScattering}, 
\item \detdesc{html/_geant4_physics_constructors_8cpp_source.html}{physics constructors} 
    combining particles and the corresponding 
    interactions, such as\\ e.g. \tts{G4OpticalPhysics},
    \tts{HadronPhysicsLHEP} or \tts{G4HadronElasticPhysics} and
\item \detdesc{html/_geant4_particles_8cpp_source.html}{predefined Geant4 physics lists}, 
    such as \tts{FTFP\_BERT},
    \tts{CHIPS} or \tts{QGSP\_INCLXX}. This option is triggered by the 
    content of the string property "extends" of the \tts{Geant4Kernel::physicsList()} action.
\end{itemize}
These constructors are internally connected to the above callbacks to register themselves. 
The constructors are instantiated using the ROOT plugin mechanism.

\noindent
The description of the above interface is only for completeness. The basic idea is,
that the physics list with its particle and physics constructors is configured
entirely data driven using the setup mechanism described in the following
chapter. However, DDG4 is not limited to the data driven approach. Specialized 
physics lists may be supplied, but there should be no need.
New physics lists could always be composed by actually providing new physics
constructors and actually publishing these using the factory methods:
\begin{code}
// Framework include files
#include "DDG4/Factories.h"

#include "My_Very_Own_Physics_Constructor.h"
DECLARE_GEANT4_PHYSICS(My_Very_Own_Physics_Constructor)
\end{code}
where \tts{My\_Very\_Own\_Physics\_Constructor} represents a sub-class of
\tts{G4VPhysicsConstructor}.

\newpage
%=============================================================================
\subsection{The Support of the Geant4 UI: \tts{Geant4UIMessenger}}
\label{sec:ddg4-user-manual-geant4action-base}
%=============================================================================

\noindent
The support of interactivity in Geant4 is mandatory to debug detector
setups in small steps. The Geant4 toolkit did provide for this reason 
a machinery of UI commands.
\begin{figure}[h]
  \begin{center}
    \includegraphics[height=70mm] {DDG4-UIMessenger}
    \caption{The design of the \tts{Geant4UIMessenger} class responsible for
        the interaction between the user and the components of \DDG and Geant4.}
    \label{fig:ddg4-tracking-action}
  \end{center}
\end{figure}

\noindent
The UI control is enabled, as soon as the property "Control" (boolean) is set to true.
Be default all properties of the action are exported.
Similar to the callback mechanism described above it is also feasible to
register any object callback invoking a method of a \tts{Geant4Action}-subclass. 

\noindent
The following (shortened) screen dump illustrates the usage of the 
generic interface any Geant4Action offers:
\begin{unnumberedcode}
Idle> ls
Command directory path : /
 Sub-directories : 
   /control/   UI control commands.
   /units/   Available units.
   /process/   Process Table control commands.
   /ddg4/   Control for all named Geant4 actions
   ...
Idle> cd /ddg4
Idle> ls
...
Control for all named Geant4 actions

 Sub-directories : 
   /ddg4/RunInit/   Control hierarchy for Geant4 action:RunInit
   /ddg4/RunAction/   Control hierarchy for Geant4 action:RunAction
   /ddg4/EventAction/   Control hierarchy for Geant4 action:EventAction
   /ddg4/GeneratorAction/   Control hierarchy for Geant4 action:GeneratorAction
   /ddg4/LCIO1/   Control hierarchy for Geant4 action:LCIO1
   /ddg4/Smear1/   Control hierarchy for Geant4 action:Smear1
   /ddg4/PrimaryHandler/   Control hierarchy for Geant4 action:PrimaryHandler
   /ddg4/TrackingAction/   Control hierarchy for Geant4 action:TrackingAction
   /ddg4/SteppingAction/   Control hierarchy for Geant4 action:SteppingAction
   /ddg4/ParticleHandler/   Control hierarchy for Geant4 action:ParticleHandler
   /ddg4/UserParticleHandler/   Control hierarchy for Geant4 action:UserParticleHandler
   ...
Idle> ls Smear1
Command directory path : /ddg4/Smear1/
 ...
 Commands : 
   show * Show all properties of Geant4 component:Smear1
   Control * Property item of type bool
   Mask * Property item of type int
   Name * Property item of type std::string
   Offset * Property item of type ROOT::Math::LorentzVector<ROOT::Math::PxPyPzE4D<double> >
   OutputLevel * Property item of type int
   Sigma * Property item of type ROOT::Math::LorentzVector<ROOT::Math::PxPyPzE4D<double> >
   name * Property item of type std::string
Idle> Smear1/show
PropertyManager: Property Control = True
PropertyManager: Property Mask = 1
PropertyManager: Property Name = 'Smear1'
PropertyManager: Property Offset = ( -20 , -10 , -10 , 0 )
PropertyManager: Property OutputLevel = 4
PropertyManager: Property Sigma = ( 12 , 8 , 8 , 0 )
PropertyManager: Property name = 'Smear1'

Idle> Smear1/Offset (200*mm, -3*mm, 15*mm, 10*ns)
Geant4UIMessenger: +++ Smear1> Setting property value Offset = (200*mm, -3*mm, 15*mm, 10*ns)  
                               native:( 200 , -3 , 15 , 10 ).
Idle> Smear1/show                                
...
PropertyManager: Property Offset = ( 200 , -3 , 15 , 10 )

\end{unnumberedcode}

\newpage

\input{sections/Setup.tex}

%=============================================================================
\section{Higher Level Components}
\label{sec:ddg4-implementation-higher-level-components}
%=============================================================================
\noindent
Layered components, which base on the general framework implement higher 
level functionality such as the handling of Monte-Carlo truth associations
between simulated energy deposits and the corresponding particles or the
generic handling of input to the simulation.

\noindent
To generalize such common behavior it is mandatory that the participating
components collaborate and understand the data components they commonly access.
The data model is shown in Figure~\ref{fig:ddg4-event-data-model}.
\begin{figure}[t]
  \begin{center}
    \includegraphics[width=120mm] {DDG4_event_data_model}
    \caption{The DDG4 event data model.}
    \label{fig:ddg4-event-data-model}
  \end{center}
\end{figure}

\noindent
{\bf{Please note}}, that this data model  is by no means to be made persistent 
and used for physics user analysis. This model is optimized to support
the simulation process and the necessary user actions to handle MC truth,
to easily and relatively fast look up and modify parent-daughter 
relationships etc. This choice is based on the assumption, that the 
additional overhead to convert particles at the input/output 
stage is small compared to the actual resource consumption of Geant4
to simulate the proper detector response.
On the other hand this choice has numerous advantages:
\begin{itemize}\itemcompact
\item Accepting the fact to convert input records allows to adapt 
  DDG4 in a simple and flexible manner to any input format. Currently 
  supported is the input from raw {\tts{LCIO}} files, {\tts{StdHep}} 
  records using {\tts{LCIO}} and {\tts{ASCII}} files using the 
  {\tts{HEPEvt}} format.
\item Similarly as for the input stage, also the output format 
  can be freely chosen by the clients.
\item No assumptions was made concerning the structure to store 
  information from energy deposits. Any information extract produced
  by the sensitive actions can be adapted provided at the output
  stage the proper conversion mechanism is present. The sensitive 
  detectors provided by DDG4 are {\bf{optional only and by no means mandatory}}.
  User defined classes may be provided at any time. Appropriate tools
  to extract MC truth information is provided at the output stage.
\item The modular approach of the action sequences described 
  in~\ref{sec:ddg4-user-manual-implementation-geant4action-sequences}
  allows to easily extend the generation sequence to handle multiple 
  simultaneous interactions, event overlay or spillover response 
  very easily~\footnote{The handling of spillover is only possible 
  if during the digitization step the correct signal shape corresponding
  to the shift of signal creation is taken into account.}
\end{itemize}

\noindent
In section~\ref{sec:ddg4-implementation-input-handling} the generic mechanism
of input data handling is described. \\
In section~\ref{sec:ddg4-implementation-particle-handling} the MC truth 
handling is discussed. \\
In section~\ref{sec:ddg4-implementation-output-handling} we describe the 
output mechanism.
\newpage

%=============================================================================
\subsection{Input Data Handling}
\label{sec:ddg4-implementation-input-handling}
%=============================================================================
\begin{figure}[t]
  \begin{center}
    \includegraphics[width=160mm] {DDG4_input_stage}
    \caption{The generic handling of input sources to DDG4.}
    \label{fig:ddg4-input-stage}
  \end{center}
\end{figure}

\noindent
Input handling has several stages and uses several modules:
\begin{itemize}\itemcompact
\item First the data structures \tts{Geant4PrimaryEvent}, 
    \tts{Geant4PrimaryInteraction} and \tts{Geant4\-Primary}\-\tts{Map} are initialized 
    by the action \tts{Geant4GenerationActionInit} 
    and attached to the {\tts{Geant4Event}} structure.
\item The initialization is then followed by any number of input modules.
  Typically each input module add one interaction. Each interaction has a 
  unique identifier, which is propagated later to all particles. Hence all
  primary particles can later be unambiguously be correlated to one of the 
  initial interactions. 
  Each instance of a \tts{Geant4InputAction} creates and fills a separate instance
  of a \tts{Geant4PrimaryInteraction}.
  In section~\ref{sec:ddg4-implementation-geant4inputaction} the functionality and
  extensions are discussed in more detail.
\item All individual primary interactions are then merged to only single record
  using the \tts{Geant4}\-\tts{Interaction}\-\tts{Merger} component.
  This components fills the \tts{Geant4PrimaryInteraction} registered to the
  \tts{Geant4Event}, which serves as input record for the next component,
\item the \tts{Geant4PrimaryHandler}. The primary handler creates the proper 
  \tts{G4PrimaryParticle} and \tts{G4PrimaryVertex} objects passed to \tts{Geant4}.
  After this step all event related user interaction with Geant4 has completed,
  and the detector simulation may commence.
\end{itemize}
All modules used are subclasses of the {\tts{Geant4}\-\tts{Generator}\-\tts{Action}} and must be
added to the \tts{Geant4}\-\tts{Generator}\-\tts{Action}\-\tts{Sequence} as described 
in~\ref{sec:ddg4-user-manual-implementation-geant4action-sequences}.

\noindent
An object of type {\tts{Geant4PrimaryEvent}} exists exactly once for 
every event to be simulated. The empty {\tts{Geant4PrimaryEvent}} is created by the
{\tts{Geant4GenerationActionInit}} component. All higher level components may then 
access the {\tts{Geant4PrimaryEvent}} object and subsequently an individual interaction
from the {\tts{Geant4Context}} using the extension mechanism as shown in 
the following code:
\begin{code}
/// Event generation action callback
void SomeGenerationComponent::operator()(G4Event* event)  {
  /// Access the primary event object from the context
  Geant4PrimaryEvent* evt = context()->event().extension<Geant4PrimaryEvent>();
  /// Access the container of interactions
  const std::vector<Geant4PrimaryEvent::Interaction*>& inter = evt->interactions();
  /// Access one single interaction to be manipulated by this component
  Geant4PrimaryInteraction* evt->get(m_myInteraction_identifier);
  ....
\end{code}
{\bf{Please note:}} To keep components simple, each component should 
only act on one interaction the component has to uniquely identify.
The identification may be implemented by e.g. an access mask passed to the 
component as a property.

%=============================================================================
\subsection{Anatomy of the Input Action}
\label{sec:ddg4-implementation-geant4inputaction}
%=============================================================================
\noindent
One input action fills one primary interaction.
\tts{Geant4InputAction} instances may be followed by decorators, which 
allow to to smear primary vertex (\tts{Geant4InteractionVertexSmear}) or
to boost the primary vertex \tts{Geant4InteractionVertexBoost} and all 
related particles/vertices.


Please note, that a possible reduction of particles in
the output record may break this unambiguous relationship between 
"hits" and particles.
......

%=============================================================================
\subsection{Monte-Carlo Truth Handling}
\label{sec:ddg4-implementation-particle-handling}
%=============================================================================
As any other component in \DDG, the   was
designed using the plugin mechanism ie. the default implementation
which was inspired by the original implementation of the MC thruth
handler developed by the Linear Collider community may easily be
overloaded.

\noindent
The Monte-Carlo thruth handler takes care that 
\begin{itemize}
\item the proper MC particles are associated with the corresponding hits
      and tracks.
\item To compress the particle record. Geant4 creates a large amount 
      of temporary particles in particluar in dense areas of the 
      detector such as calorimeters. In calorimeters however, the 
      hits within a confined volume should be assigned to the incoming
      track. In addition a track is only supposed to be kept if it 
      satisfies certain criteria.
\end{itemize}
To achieve this functionality the Monte-Carlo thruth handler implemented
in the class \tts{Geant4ParticleHandler} firstly
\begin{itemize}
\item implements the interface \tts{Geant4MonteCarloTruth} which gets
      called whenever an interaction occurs in a sensitive volume
      which is modeled by an instance of a instance of 
      \tts{Geant4SensitiveAction}.
\item to properly manager the MC particle records the 			
	  \tts{Geant4ParticleHandler} either inherits or uses
	  the callbacks provided by the DDG4 interfaces to the
	  \begin{itemize}
		  \item \tts{Geant4GeneratorAction}
		  \item \tts{Geant4EventAction}
		  \item \tts{Geant4TrackingAction}
		  \item \tts{Geant4SteppingAction}.
	  \end{itemize}
	  While the response of one track is simulated, all relevant 
	  information is extracted in the callbacks and at the end of the
	  simulation of the track response a decision is taken whether to
	  store the information of the Geant4 track in the MC particle
	  record or not.
\item A Geant4 track is saved in the MC track record if
	  \begin{itemize}
	  	  \item the track did not intercat with the detector, but 
	  	  		is part of the Monte-Carlo record originating
	  	  		from the original generator consisting of quarks,
	  	  		leptons, gluons, gammas etc.
		  \item the track was declared to Geant4 as a Geant4 primary
		        track from the generator action. These are either 
		        long-living remnants of the underlying hard interaction
		        of particles decaying macroscopically inside the
		        experiment volume like e.g. B-mesons.
		  \item the track exits the world volume.
		  \item the track is mother particle to secondaries.
		  \item the track created a hit in a \it{"tracker"}-type
		        sensitive volume.
		  \item the track is above a certain energy threshold and
		  		has at least one associated hit either in a 
		  		\it{calorimter}-type volume of a \it{tracker}-type
		  		volume.
	  \end{itemize}
	  For all tracks purged from the MC particle record, any resulting
	  energy deposit is associated to the last parent particle 
	  stored in the MC particle record.
\item To fine-tune the Monte-Carlo truth handler in \DDG a 
	  use class with interface  \tts{Geant4UserParticleHandler} 
	  may be supplied, which allows to customize and fine tune
	  if a given MC particle is supposed to be kept in the final
	  record or not. This user class receives the identical callbacks
	  as the truth handler, but at the end of the simulation of each
	  track (the end-tracking-action) a call is issued by the truth
	  handler and allows to override the decision whether to keep
	  or dismiss storing a track.
\end{itemize}

\noindent
As mentioned above this implementation is only an example how
to realize such a Monte-Carlo truth logic. It is assumed that the
interface \tts{Geant4ParticleHandler} together with the easy-to-use
subscription mechanism to all callbacks provided by Geant4
allow to easily implement other Monte-Carlo truth mechanisms.

\vspace{0.3cm}
\noindent
The following table shows all properties accepted by the 
\DDG Monte-Carlo truth handler.

\vspace{0.3cm}
\noindent
\begin{tabular}{ l p{10cm} }
\hline
\bold{Class name}      & \tts{Geant4ParticleHandler}           \\
\bold{File name}       & \tts{DDG4/src/Geant4ParticleHandler.cpp} \\
\bold{Type}            & \tts{Geant4Action}                                  \\
\hline 
\bold{Component Properties:}   & defaults apply                              \\
\bold{PrintEndTracking} (bool) & \tts{Extra printout at the end of the } \\
                               & \tts{tracking action for debugging} \\
\bold{PrintStartTracking} (bool) & \tts{Extra printout at the start of the } \\
                               & \tts{tracking action for debugging} \\
\bold{KeepAllParticles} (bool) & \tts{Flag to override any NC particle removal} \\
\bold{SaveProcesses} (bool)    & \tts{Save all produces of the specified} \\
                               & \tts{Geant4 particle processes} \\
\bold{MinimalKineticEnergy} (bool)  & \tts{Minimal energy cut required to accept a MC particle} \\
\bold{MinDistToParentVertex} (bool) & \tts{Minimal distance to the parent's }\\
                               & \tts{start-vertex in order to become an independent particle} \\
                               & \tts{Used to e.g. suppress Delta-rays} \\
\end{tabular}

\newpage

%=============================================================================
\section{Output Data Handling}
\label{sec:ddg4-implementation-output-handling}
%=============================================================================

\noindent
The output of the data record of the accepted MC particle record
and the corresponding sets of hits in the various subdetectors is 
basic to further handing data originating from simulated particle
collisions. In \DDG the handling of output data is implemented as
a specialization of a \tts{Geant4EventAction} since the output
needs to written at the end of each simulated event.

\noindent
Currently there are three types of output formats implemented:
\begin{itemize}
\item Writing the MC particle record and the Geant4 hits
	  natively as ROOT objects to a ROOT file.
	  This is a very simple solution, writes the entire event
	  as a ROOT TTree object. The persistent data format of the 
	  objects is the same as the transition data format in memory
	  used during the simulation step.
\item Writing the particle record and the hit structures in LCIO 
      data format. For details of the LCIO data format
      please consult the LCIO manual.
\item Writing the particle record and the hit structures in 
      the EDMS data format developed by the CERN/SFT data format.
      For details of the LCIO data format please consult the LCIO
      manual.
\end{itemize}

Unless the native ROOT format is used for data output,
the data format of the transient representation 
of Monte-Carlo particles and the resulting tracker- and
calorimeter hits differes from the persistent representation 
and requires data conversion. The overheads of such conversions however
are typically neglidgeble with respect to the rather large resource
usage required for simulation.

\vspace{0.3cm}
\noindent
The component properties of the generic output class:

\vspace{0.3cm}
\noindent
\begin{tabular}{ l p{10cm} }
\hline
\bold{Class name}      & \tts{Geant4OutputAction}           \\
\bold{File name}       & \tts{DDG4/src/Geant4OutputAction.cpp} \\
\bold{Type}            & \tts{Geant4Action}                                  \\
\hline 
\bold{Component Properties:}   & defaults apply                              \\
\bold{Output} (string) & \tts{String representation of the output-file} \\
\bold{HandleErrorsAsFatal} (bool) & \tts{Convert any error of the concrete implementation}\\
                                  & \tts{into a fatal exception causing \DDG to stop processing.}\\
\end{tabular}

\vspace{0.3cm}
\noindent
The component properties of the ROOT output class:

\vspace{0.3cm}
\noindent
\begin{tabular}{ l p{10cm} }
\hline
\bold{Class name}      & \tts{Geant4Output2ROOT}           \\
\bold{File name}       & \tts{DDG4/src/Geant4Output2ROOT.cpp} \\
\bold{Type}            & \tts{Geant4Action}                                  \\
\hline 
\bold{Component Properties:}   & defaults apply                              \\
\bold{Section} (string)        & \tts{Name of the ROOT TTree to store the event data.}\\
                               & \tts{Default: EVENT} \\
\bold{HandleMCTruth} (bool)    & \tts{Handle the results of the Monte-Carlo thruth handler}\\
                               & \tts{when outputting data}\\
\bold{DisabledCollections}     & \tts{vector<string>}\\
                               & \tts{Geant4 filled collections, which should be excluded}\\
                               & \tts{from the output record.}\\
\bold{DisableParticles} (bool) & \tts{Inhibit the output of the particle record.}
                               
\end{tabular}


\newpage

\input{sections/MT.tex}
\input{sections/Components.tex}

%=============================================================================
\newpage
\begin{thebibliography}{9}
\bibitem{bib:DD4hep}  DD4Hep web page, http://aidasoft.web.cern.ch/DD4hep.

\bibitem{bib:LHCb} 		LHCb Collaboration, 
                "LHCb, the Large Hadron Collider beauty experiment, reoptimised detector 
				design and performance", CERN/LHCC 2003-030

\bibitem{bib:LHCb-geometry} S. Ponce et al., 
                "Detector Description Framework in LHCb", 
                International Conference on Computing in High Energy and Nuclear Physics  (CHEP 2003), 
                La Jolla, CA, 2003, proceedings. 

\bibitem{bib:ILD}  The ILD Concept Group, 
                   "The International Large Detector: Letter of Intent",\\
                   ISBN 978-3-935702-42-3, 2009.

\bibitem{bib:SiD}  H. Aihara, P. Burrows, M. Oreglia (Editors),
                   "SiD Letter of Intent",
                   arXiv:0911.0006, 2009.

\bibitem{bib:ROOT-tgeo} R.Brun, A.Gheata, M.Gheata, "The ROOT geometry package",\\
                    Nuclear Instruments and Methods {\bf{A}} 502 (2003) 676-680.

\bibitem{bib:ROOT} R.Brun et al., 
                   "Root - An object oriented data analysis framework",\\
                    Nuclear Instruments and Methods {\bf{A}} 389 (1997) 81-86.

\bibitem{bib:geant4}  S. Agostinelli et al., 
                   "Geant4 - A Simulation Toolkit", \\
                    Nuclear Instruments and Methods {\bf{A}} 506 (2003) 250-303.

\bibitem{bib:LCDD} T.Johnson et al., 
                   "LCGO - geometry description for ILC detectors", 
                   International Conference on Computing in High Energy and Nuclear Physics  (CHEP 2007), 
                   Victoria, BC, Canada, 2012, Proceedings.

\bibitem{bib:lcsim} N.Graf et al., 
                   "lcsim: An integrated detector simulation, 
                   reconstruction and analysis environment", 
                   International Conference on Computing in High Energy and Nuclear Physics (CHEP 2012),
                   New York, 2012, Proceedings.

\bibitem{bib:GDML} R. Chytracek et al.,
                   "Geometry Description Markup Language for Physics Simulation and Analysis
                   Applications",
                   IEEE Trans. Nucl. Sci., Vol. 53, Issue: 5, Part 2, 2892-2896,
                   http://gdml.web.cern.ch.

\bibitem{bib:DDSegmentations} C.Grefe et al.,
                   "The DDSegmentation package", 
                   Non existing documentation to be written.
\bibitem{bib:Geant4-multi-threading} Geant4 Multi threading Guides. 
	Please see for details:\\
	https://twiki.cern.ch/twiki/bin/view/Geant4/Geant4MTAdvandedTopicsForApplicationDevelopers,\\
	https://twiki.cern.ch/twiki/bin/view/Geant4/QuickMigrationGuideForGeant4V10,\\
	http://geant4.slac.stanford.edu/tutorial/MC2015G4WS/Multithreading.pdf

\end{thebibliography}
%=============================================================================
\end{document}
